% !TEX root = msri731.tex

\section{The weight-monodromy conjecture}
\thanksauthor{Ana Caraiani (Feb.\ 18)}




The goal is to understand Scholze's proof of the weight-monodromy conjecture 
for hypersurfaces in projective space. We will do this by comparing \'etale 
topoi in characteristic zero and characteristic $p$. 





\subsection{Tilting}

Recall the tilting equivalence. Let $K$ be a perfectoid field (i.e. a complete 
non-archimedean field with non-discrete valuation such that Frobenius 
$K^\circ/\pi \to K^\circ/\pi$ is surjective, where $\pi\in K^\circ$ has 
$|p|\leqslant |\pi| <1$). The tilt of $K$ is $K^\flat$, which as a set is 
\[
  K^\flat = \varprojlim_{x\mapsto x^p} K .
\]
There is a section map $(-)^\sharp:K^\flat \to K$ that is multiplicative and 
continuous, but not additive or surjective. We say that 
$\spa(K,K^\circ)$ tilts to $\spa(K^\flat,K^{\flat\circ})$. A baby case of the 
almost purity theorem tells us that there is an equivalence of categories 
$K_\fet \simeq K^\flat_\fet$. This leads to an isomorphism of Galois groups 
$\gal(\bar K/K) \simeq \gal(\bar K^\flat/K^\flat)$. 

This can be generalized to perfectoid spaces over $K$. Recall that a 
\emph{perfectoid space} over $K$ is an adic space $X$ over $K$ that is locally 
of the form $\spa(R,R^+)$, where $R$ is a perfectoid $K$-algebra and 
$R^+\subset R^\circ$ is an open integrally closed subring. The space $X$ tilts 
(locally) to $\spa(R^\circ,R^{\circ +})$, yielding a functor $X\mapsto X^\flat$ 
from perfectoid spaces over $K$ to perfectoid spaces over $K^\flat$. 

Recall that we have a map $R^\flat = \varprojlim_{x\mapsto x^p} R \to R$ given 
by $(x_i)_{i\geqslant 1} \mapsto x_0$. This map is continuous, multiplicative, 
but not surjective. However, it induces an isomorphism 
$R^+/\pi \simeq R^{\flat +}/\pi^\flat$. This lets define a homeomorphism 
$|X|\to |X^\flat|$, $x\mapsto x^\flat$, where for $f\in R^\flat$, we have 
$|f(x^\flat)| = |f^\sharp(x)|$. This homeomorphism preserves rational 
subsets. To prove this, one uses an approximation lemma: given $f\in R^+$, 
there exists $g\in R^{\flat +}$ such that $|f|+\|g^\sharp|$ except when both 
are ``very small'' (i.e. except when $|f|<\varepsilon$, in which case we only 
require $|g^\sharp|<\varepsilon$). 

The structure sheaves $(\sO_X,\sO_X^+)$ of $X$ tilt to sheaves 
$(\sO_{X^\flat}, \sO_{X^\flat}^+)$ on $X^\flat$. Recall that an \'etale 
morphism of perfectoid spaces is a composite of a finite \'etale morphism with 
an open immersion. Using the almost purity theorem, we get an isomorphism 
$X_\et\simeq X^\flat_\et$. Recall that a key point here is that locally, 
$\widehat{\sO_{X,x}^+}[\frac 1 \pi] \simeq \widehat{k(x)}$, and both are perfectoid 
fields. This reduces the proof of the almost purity theorem to the case for 
perfectoid fields. This isomorphism on stalks actually holds more generally 
for analytic adic spaces. The sheaf property is crucial here. 





\subsection{Comparing perfectoid spaces with noetherian spaces}

To apply any of this to the weight-monodromy conjecture, we need to compare 
\'etale sites and underlying spaces of perfectoid spaces with the \'etale 
sites and underlying spaces of locally noetherian adic spaces. For example, 
let $(\dP_K^n)^\ad$ be the adic projective space glued out of 
$\spa(K\langle T_1,\dots,T_n\rangle,K^\circ\langle T_1,\dots,T_n\rangle)$ in 
the usual way. Also define $(\dP_K^n)^\perf$ by glueing copies of 
\[
  \spa(K\langle T_1^{1/p^\infty},\dots,T_n^{1/p^\infty}\rangle,K^\circ\langle T_1^{1/p^\infty},\dots,T_n^{1/p^\infty}\rangle)
\] 
in the usual way. 

\begin{definition}
Let $X$ be a perfectoid space over $K$, and $\{X_i\}$ a filtered inverse 
system of noetherian adic spaces over $K$, together with a compatible system 
of maps $\varphi_i :X\to X_i$. We say that $X\sim \varprojlim X_i$ if 
\begin{enumerate}
  \item The induced map $|X|\to \varprojlim |X_i|$ is a homeomorphism. 
  \item For all $x\in X$ inducing $x_i\in X_i$, the map 
    $\varinjlim k(x_i) \to k(x)$ has dense image. 
\end{enumerate}
\end{definition}

Note that if $Y\to X_i$ is an \'etale morphism of adic spaces, then 
$Y\times_{X_i} X\sim \varprojlim_{j\geqslant i} Y\times_{X_i} X_j$. 

Consider $\dP_K^{n,\perf}$ as perfectoid space over $K$. 

\begin{theorem}
We have $\dP_K^{n,\perf,\flat} \simeq \dP_{K^\flat}^{n,\perf}$. 

2. $\dP_K^{n,\perf} \sim \varprojlim_\Phi \dP_K^{n,\ad}$, where 
$\Phi(x_0:\cdots:x_n) = (x_0^p:\cdots:x_n^p)$. 

3. There are homeomorphisms of topological spaces 
$|\dP_{K^\flat}^{n,\ad}| \simeq |\dP_{K^\flat}^{n,\perf}|\simeq |\dP_K^{n,\perf}| \simeq \varprojlim_\Phi |\dP_K^{n,\ad}|$. 

4. There is an equivalence of \'etale topoi 
$\dP_{K^\flat,\et}^{n,\ad,\sim}\simeq \varprojlim_\Phi \dP_{K,\et}^{n,\ad,\sim}$. 

5. Denote by $\pi$ the map 
$|\dP_{K^\flat}^{n,\ad} \to |\dP_K^{n,\ad}|$. If 
$U\subset |\dP_K^{n,\ad}|$ is open,  put $V=\pi^{-1}(U)\subset |\dP_{K^\flat}^{n,\ad}|$.
There is a commutative diagram 
\[\xymatrix{
  V_\et^\sim \ar[r] \ar[d] 
    & \dP_{K^\flat,\et}^{n,\ad,\sim} \ar[d] \\
  U_\et^\sim \ar[r] 
    & \dP_{K,\et}^{n,\et,\sim} 
 }\] 
\end{theorem}
\begin{proof}
1. It suffices to check this on affinoid pieces, where this comes down to the 
fact that 
\[
  (K^\circ/\pi)\langle T_1^{1/p^\infty},\dots,T_n^{1/p^\infty}\rangle = (K^{\flat \circ}/\pi^\flat) \langle T_1^{1/p^\infty},\dots,T_n^{1/p^\infty}\rangle . 
\]

3. This follows from part 2 and the tilting equivalence. 
\end{proof}

Note that part 3 tells us that $|\dP_{K^\flat}^{n,\ad}|\simeq \varprojlim_\Phi |\dP_K^{n,\ad}|$. 

There is a map $\varprojlim_\Phi |\dP_K^{n,\ad}| \to |\dP_K^{n,\ad}|$ that on 
coordinates is $(x_0:\cdots:x_n)\mapsto (x_0^\sharp:\cdots : x_n^p)$. 

\begin{corollary}
There are natural isomorphisms 
\[
  \h^i(\dP_{K^\flat,\et}^{n,\ad},\dZ/\ell^m) \simeq \h^i(\dP_{K,\et}^{n,\ad}.\dZ/\ell^m) 
\]
\end{corollary}
In proving the corollary, one can assume that $K$ (and hence $K^\flat$) are 
separably closed. 





\subsection{Proving the weight-monodromy conjecture}

Let $k$ be a local field, $X$ a proper smooth variety over $k$. Then the groups 
$V=\h^i(X_{\bar k},\bar\dQ_\ell)$ admit a continuous action of the group 
$\gal(\bar k/k)$. Assume $\ell\ne p$. Then this action is characterized by the 
action a (lift of) Frobenius. Let $\pi\in k$ be a uniformizer and 
$\dF_q=k^\circ/\pi$. By the Grothendieck $\ell$-adic monodromy theorem, after a 
finite base-change, everything is defined in terms of a nilpotent operator 
$N$ (which encodes the action of the tame inertia). 

From basic linear algebra, we know that we can write 
$V=\bigoplus_{j=0}^{2 i} V_i$, where on each $V_i$ the eigenvalues of Frobenius 
are Weil numbers of weight $j$ (i.e. their absolute value is 
$q^{j/2}$ for each complex embedding $\bar\dQ_\ell\hookrightarrow \dC$). If 
$X$ has good reduction, we know that only one weight occurs, so $N=0$. 
This is because in general, $N:V_j \to V_{j-2}$. 

\begin{conjecture}[Deligne]
If $V$ is as above, then $N^j:V_{i+j} \to V_{i-j}$ is an isomorphism for all 
$0\leqslant j\leqslant i$. 
\end{conjecture}

Equivalently, we can define the monodromy filtration on $V$ which is uniquely 
characterized by the conclusion of the conjecture, and the conjecture claims 
that the monodromy filtration and weight filtration are the same. 

The weight-monodromy conjecture is known in the equal-characteristic case. 

\begin{theorem}[Deligne]
Let $C$ be a curve over $\dF_q$, $x\in C(\dF_q)$ such that 
$k$ is the local field of $C$ at $x$. If $X\to C\smallsetminus \{x\}$ is 
smooth, then $X_k=X\times_{C\smallsetminus \{x\}} \spec(k)$ satisfies the 
weight-monodromy conjecture. 
\end{theorem}

Note that the weight-monodromy conjecture only concerns the monodromy operator 
and the Frobenius operator. If we let $K=\widehat{k(\pi^{1/p^\infty})}$, then these 
remain the same, so it suffices to work with varieties over $K$. 

Let $Y\subset \dP_K^n$ be a smooth hypersurface. We can pass to the adic world 
to obtain $Y^\ad\subset \dP_K^{n,\ad}$. There is a comparison theorem between 
the \'etale cohomology of $Y$, and the \'etale cohomology of an open adic 
neighborhood $\widetilde Y$ of $Y^\ad$. Over $\dC_p$, we have a diagram 
\[\xymatrix{
  \dP_{\dC_p^\flat}^n \ar[r] 
    & \widetilde Y_{\dC_p} \\
  \pi^{-1}(\widetilde Y) \ar@{^{(}->}[u] 
    & Y_{\dC_p} \ar@{^{(}->}[u] 
}\]

\begin{lemma}
Suppose $Y$ is cut out by a homogeneous polynomial $f$ of degree $d$. Then 
there exists a homogeneous polynomial 
$g\in \dC_p^\flat\langle T_1^{1/p^\infty},\dots,T_n^{1/p^\infty}\rangle$ such 
that $|f(x)|\leqslant \epsilon$ if and only if $|g^\sharp(x)|\leqslant \epsilon$ 
for all $x\in \dP_{\dC_p}^{n,\ad}$. 
\end{lemma}

It turns out that $\pi^{-1}(\widetilde Y)$ will contain some $Z$ algebraic 
over $\dC_p$. Huber gives comparison isomorphisms 
$\h^i(Y,\dZ/\ell^m)\simeq \h^i(\widetilde Y,\dZ/\ell^m)$ for $\widetilde Y$ 
sufficiently small. In this setting, we get an injection   
\[
  \h^i(Y_{\dC_p,\et},\overline\dQ_\ell) \to \h^i(Z_{\dC_p,\et},\overline\dQ_\ell)
\]
that is $\gal(\bar K/K)\simeq \gal(\bar K^\flat/K^\flat)$-equivariant and 
compatible with cup product. 

One can choose $Z$ so that the representation on the right satisfies the 
weight-monodromy conjecture, so it suffices to show that the space on the left 
is a direct summand. We use this using Poincar\'e duality and the following 
lemma. 

\begin{lemma}
Let $f\in K^\circ\langle T_0^{1/p^\infty},\dots,T_n^{1/p^\infty}\rangle$ be a 
homogeneous polynomial of degree $d$. Then for all $\varepsilon,c>0$, there 
exists $g\in K^{\flat\circ}\langle T_0^{1/p^\infty},\dots,T_n^{1/p^\infty}\rangle$ 
such that $|f(x)-g^\sharp(x)|\leqslant |\pi|^{1-\varepsilon} \cdot \max(|\pi|^c,|f(x)|)$. 
\end{lemma}
\begin{proof}
Use induction on $c$. Let $R^\circ=K\langle T_0^{1/p^\infty},\dots,T_n^{1/p^\infty}\rangle$. 
We know that $R^\circ/\pi\simeq R^{\flat\circ}/\pi^\flat$. Almost mathematics tells 
us that if $|f(x)|\leqslant |\pi|^c$, then $f-g\in |\pi|^{1-\varepsilon+c} \sO_X^+(U)$. 
\end{proof}

To conclude, we get a map $\h^i(Y,\overline\dQ_\ell)\to \h^i(Z,\overline\dQ_\ell)$. 
To show this is a direct summand, it is enough to look at $i=2\dim Y$, where the 
map is either an isomorphism or zero. If the map is zero, we get that 
$\h^i(\dP_{\dC_p^\flat}^n,\overline\dQ_\ell) \to \h^i(Z_{\dC_p^\flat},\overline\dQ_\ell)$ 
is zero, but this cannot happen (because of Chern classes). Thus the top-degree 
map is an isomorphism, so we use compatibility with cup-products and Poincar\'e 
duality to get the result. 
