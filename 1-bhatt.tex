\documentclass{article}

\usepackage{msri-style}

\title{Almost ring theory 1}
\author{Bhargav Bhatt}
\date{February 17, 2014}

\begin{document}
\maketitle





Recall that there is a ``tilting correspondence'' relating objects in characteristic 
zero to objects in characteristic $p$. This correspondence works not at an integral 
level, but at an ``almost integral level,'' i.e. using the language of almost 
mathematics. A good general reference for almost mathematics is the book \cite{gr03} 
of Gabber and Ramero. 





\section{Almost mathematics}

Throughout, a \emph{non-archimedean field} $K$ is a field equipped with a 
rank-one valuation $|\cdot|:K^\times \to \dR_{>0}$. If $K$ is any 
non-archimedean field, $K^\circ$ denotes the valuation ring 
$\{x\in K:|x|\leqslant 1\}$ of $K$. 

\begin{definition}
A \emph{perfectoid field} $K$ is a complete non-archimedean field such that: 
\begin{enumerate}
  \item The residue characteristic of $K$ is $p>0$. 
  \item The associated rank-one valuation is non-discrete. 
  \item The Frobenius map $\Phi:K^\circ/p \to K^\circ/p$ is surjective. 
\end{enumerate}
\end{definition}

\begin{example}
The $p$-adic completion of $\dQ_p(p^{1/p^\infty})$ is perfectoid. 
\end{example} 

\begin{example}
The $p$-adic completion $\dC_p$ of $\overline\dQ_p$ is perfectoid. 
\end{example}

\begin{example}
If $k$ is a non-archimedean field of positive characteristic, then the 
``perfectification'' of $k$ is perfectoid. Here the perfectification of 
$k$ is the completion of the colimit $\varinjlim_\Phi k$, where $\Phi$ is 
Frobenius. The perfectification of $\dF_p\lau t$ is the $t$-adic completion of 
$\dF_p\lau t(t^{1/p^\infty})$. 
\end{example}

\begin{example}
The field $\dQ_p$ is not perfectoid because its valuation is discrete. 
\end{example}

Let $K$ be a perfectoid field, and denote by $\fm\subset K^\circ$ the unique 
maximal ideal. Since the valuation on $K$ is non-discrete, $\fm$ is not 
finitely generated. One has $\fm\otimes \fm \xrightarrow\sim \fm = \fm^2$, the 
first isomorphism coming from the flatness of $\fm$. Let 
$\Sigma\subset K^\circ\Mod$ be the full subcategory consisting of $\fm$-torsion 
modules. Since $\fm^2=\fm$, $\Sigma$ is a (thick) abelian Serre subcategory. 

\begin{definition} 
Let $K$ be a perfectoid field. 
\begin{enumerate}
  \item A $K^\circ$-module $M$ is \emph{almost zero} if $M\in\Sigma$, i.e. $\fm M=0$. 
  \item The category of \emph{$K^{\circ a}$-modules} is the Serre quotient 
    $K^\circ\Mod / \Sigma$. 
\end{enumerate}
\end{definition}

For example, the residue field $K^\circ/\fm$ is almost zero. On the other hand, 
$K^\circ/p$ (or $K^\circ / t$ for any $t\in \fm$) is not almost zero.

Write $(-)^a$ for the localization functor $K^\circ\Mod \to K^{\circ a}\Mod$. 
A crucial fact is that $(-)^a$ has both left and right adjoints, denoted 
$N\mapsto N_!$ and $N\mapsto N_\ast$, respectively. This allows us to easily 
compute hom-sets in $K^{\circ a}\Mod$. Indeed, 
\begin{align*}
  \hom_{K^\circ}(M_!,N) &= \hom_{K^{\circ a}}(M^a,N^a) = \hom_{K^\circ}(M,N_\ast)
\end{align*}
so the problem of computing $\hom_{K^{\circ a}}(M^a,N^a)$ is reduced to that 
of computing $M_!$ and $N_\ast$. Fortunately, the functors $(-)_!$ and 
$(-)_\ast$ can be explicitly described. One has 
\begin{align*}
  (T^a)_! &= \fm\otimes T \\
  (T^a)_\ast &= \hom_{K^\circ}(\fm,T) .
\end{align*}
For a $K^{\circ a}$-module $M$, one calls $M_\ast$ the module of ``almost 
elements'' of $M$. 

This notation is motivated by topology. If $j:U\hookrightarrow X$ is the 
inclusion of an open subset into a topological space, then the restriction 
functor $j^\ast:\sh(X) \to \sh(U)$ has left and right adjoints 
$j_!$ and $j_\ast$. This suggests that $K^{\circ a}\Mod$ is the category 
of sheaves on some subscheme of $\spec(K^\circ)$ that lies ``in between'' 
the special point and generic fiber, even though no such subscheme exists. 

It is an easy exercise (and purely formal) to prove that $N\mapsto N_!$ is an 
exact functor. Moreover, if $N$ is a $K^{\circ a}$-module, then 
$(N_!)^a = N = (N_\ast)^a$, from which we see that $N\mapsto N_!$ and 
$N\mapsto N_\ast$ are fully faithful. (This too is an exercise in pure category 
theory.) 

While $(-)_!$ and $(-)_\ast$ are sections of $(-)^a$, one does not generally 
have $(M^a)_!=M$ or $(M^a)_\ast=M$, for $M$ a $K^\circ$-module. For example, if 
$M=K^\circ$, then $(M^a)_! = \fm\ne M$. Similarly, if $M=\fm$, then 
$(M^a)_\ast = K^\circ\ne M$. 

The subcategory $\Sigma\subset K^\circ\Mod$ is an ideal in the sense that it is 
closed under taking tensor products with arbitrary $K^\circ$-modules. Thus 
$K^{\circ a}\Mod$ inherits a tensor product from $K^\circ\Mod$, so we can talk 
about algebras in $K^{\circ a}\Mod$. For example, any $K^\circ$-algebra $A$ 
induces a $K^{\circ a}$-algebra $A^a$. Using the functor $(-)_\ast$, one can 
show that every $K^{\circ a}$-algebra is of this form. Moreover, with the 
obvious definition of a module over a $K^{\circ a}$-algebra $A^a$, one sees that 
all $A^a$-modules are of the form $M^a$, for $M$ an $A$-module. 

The abelian tensor category $(K^{\circ a}\Mod,\otimes)$ has an internal 
hom-functor. For $K^{\circ a}$-modules $M,N$, the hom-set 
$\hom_{K^{\circ a}}(M,N)$ is naturally a $K^\circ$-modules, so we can put 
\[
  \alhom(M,N) = \hom_{K^{\circ a}}(M,N)^a .
\]





\section{Almost commutative algebra}

As before, let $K$ be a perfectoid field. 

\begin{definition}
Let $A$ be a $K^{\circ a}$-algebra, $M$ an $A$-module. 
\begin{enumerate}
  \item $M$ is \emph{flat} if $M\otimes_A -$ is an exact functor. 
  \item $M$ is \emph{almost projective} if the functor $\alhom(M,-)$ is exact. 
  \item Assume $A=R^a$ and $M=N^a$. Then $M$ is \emph{almost finitely generated} 
    if for all $\epsilon\in \fm$, there exists a finitely generated 
    $R$-module $N_\epsilon$ with a map $f_\epsilon:N_\epsilon\to M$ such that 
    $\ker(f_\epsilon)$ and $\coker(f_\epsilon)$ are killed by $\epsilon$. 
  \item If the number of generaters of the $N_\epsilon$ can be taken to be 
    bounded, we say that $M$ is \emph{uniformly finitely generated}
\end{enumerate}
\end{definition}
There is an obvious notion of an \emph{almost finitely presented} $A$-module 
along the same lines. 

Let $A=R^a$ be an almost $K^{\circ a}$-algebra, and suppose $M=N^a$. Then $M$ is 
almost flat if and only if $\tor_i^R(N,-)$ takes values in almost zero 
modules for all $i>0$. Similarly, $M$ is almost projective if and only if 
$\ext_R^i(M,-)$ takes values in almost zero modules for all $i>0$. 

It is not the case that an almost $A$-module is almost projective if and only 
if it is projective in the categorical sense. In fact, it is a good exercise to 
show that if $M$ is a $K^{\circ a}$-module that is projective, then $M=0$. 

Any finitely generated ideal $I\subset K^\circ$ is an almost finitely generated 
$K^{\circ a}$-module. In fact, such $I$ are uniformly almost finitely generated. 
Fix $r\in \dR_{>0}$ such that $r\notin |K^\times|$. Then the ideal 
$I_r=\{f\in K^\circ:|f|<r\}$ is not finitely generated, but is uniformly almost 
finitely generated. 






\section{Unramified and etale morphisms}

Suppose $A\to B$ is a finite \'etale map of commutative rings. Then the 
diagonal $\spec(B)\hookrightarrow \spec(B\otimes_A B)$ is clopen. It follows 
that there is a unique idempotent $e\in B\otimes_A B$, called the \emph{diagonal 
idempotent} such that 
\begin{enumerate}
  \item $e^2=e$ 
  \item $\mu(e)=1$, where $\mu:B\otimes_A B\to B$ is the multiplication map 
  \item $\ker(\mu)\cdot e = 0$. 
\end{enumerate}

For example, if $A\to B$ is Galois with group $G$, then 
$B\otimes_A B\simeq \prod_{g\in G} B$ via 
$(b_1\otimes b_2)\mapsto (b_1\cdot g(b_2))_{g\in G}$. In this setting, $e$ is the 
element $(1,0,\dots,0)$. 

If one writes $e=\sum_{i=1}^N x_i\otimes y_i$ with $x_i,y_i\in B$, then 
$\tr(e) = \sum \tr(x_i y_i) = 1$. Moreover, consider the maps 
$\alpha:B\to A^{\oplus N}$ and $\beta:A^{\oplus N} \to B$ defined by 
\begin{align*}
  \alpha(b) &= (\tr(b x_i))_i \\
  \beta(a_i)_i &= \sum_i a_i y_i .
\end{align*}
One has $\beta\circ\alpha = 1_B$, so $B$ is projective. 




\section{Almost \'etale extensions}

As before, let $K$ be a perfectoid field. 

\begin{definition}
Let $f:A\to B$ be a morphism of $K^{\circ a}$-algebras. 
\begin{enumerate}
  \item $f$ is \emph{unramified} if there exists $e\in (B\otimes_A B)_\ast$ 
    such that $e^2=e$, $\mu(e)=1$, and $\ker(\mu)\cdot e=0$. 
  \item $f$ is \emph{\'etale} if it is flat and unramified. 
  \item $f$ is \emph{finite \'etale} if it is \'etale and $B$ is an almost finitely 
presented projective $A$-module. 
\end{enumerate}
\end{definition}

If $A$ is a $K^{\circ a}$-algebra, write $A_\fet$ for the category of finite 
\'etale $A$-algebras. There is a good deformation theory for finite \'etale 
extensions. For example, if $I\subset A$ is nilpotent, then the natural functor 
$A_\fet \to (A/I)_\fet$ is an equivalence of categories. 

Suppose $K$ is a perfectoid field of characteristic $p>0$. Choose an element 
$0\ne t\in\fm$. Let $A$ be a flat $K^\circ$-algebra which is integrally 
closed inside $A[1/t]$. Let $B'$ be a finite \'etale $A[1/t]$-algebra. Let $B$ 
be the integral closure of $A$ in $B'$. 

\begin{proposition}
If $A$ is perfect, then $A^a \to B^a$ is finite \'etale as a map of 
$K^{\circ a}$-allgebras. 
\end{proposition}
\begin{proof}
We have a diagonal idempotent $e\in B'\otimes_A B'$. There exists 
some $N>0$ such that $t^N\cdot e\in B\otimes_A B$. Everything in sight is 
perfect, so we can apply the inverse of Frobenius to conclude that 
$(t^N)^{1/p^n}\cdot e\in B\otimes_A B$ for all $n>0$. Thus  
$e\in (B\otimes_A B)$ is almost integral, so $B$ is almost \'etale. 

To see that $B$ is almost finitely presented, fix $\epsilon\in\fm$. Write 
$\epsilon\cdot e = \sum_{i=1}^N x_i\otimes y_i$. The composite 
$B\to A^{\oplus N} \to B$ of $b\mapsto (\tr(b x_i))_i$ and 
$(a_i)\mapsto \sum a_i y_i$ is multiplication by $\epsilon$, so $B$ is uniformly almost 
finitely presented projective. 
\end{proof}





\bibliographystyle{alpha}
\bibliography{msri-sources}

\end{document}
