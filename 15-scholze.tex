% !TEX root = msri731.tex

\section{Future directions 1: formal \texorpdfstring{$\dQ_p$}{Qp}-vector spaces of slope \texorpdfstring{$>1$}{>1}}
\thanksauthor{Peter Scholze (Feb.\ 20)}





As in my last talk, there will be many ``open problems,'' i.e. basic questions 
about perfectoid spaces whose answer is not known. Much of this is motivated by 
discussions with Fargues and Weinstein. 





\subsection{Crystals and Dieudonn\'e modules}

Let $k$ be a perfect field. Let $\cO_{K_0}=W(k)$ be the ring of $p$-typical Witt 
vectors over $k$, and let $K_0=\cO_{K_0}[\frac 1 p]$. Let $G$ be a $p$-divisible 
group over $k$. 

\begin{theorem}
There is an anti-equivalence of categories:
\[
  \{\text{$p$-divisible groups over $k$}\}^\circ \simeq \left\{\begin{array}{c} (M,F,V) :\text{ $M$ is a finite free $\cO_{K_0}$-module} \\ \text{$F:M\to M$ is $\sigma$-linear, $V:M\to M$ is $\sigma^{-1}$-linear} \\ \text{and $V F = F V=p$}\end{array}\right\}
\]
\end{theorem}

\begin{definition}
An \emph{isocrystal} is a finite free $K_0$-vector space $N$ together with 
$\varphi:N\to N$, a $\sigma$-linear automorphism. 
\end{definition}

\begin{theorem}[Dieudonn\'e-Manin]
Assume $k$ is algebraically closed. Then any isocrystal $(N,\varphi)$ is isomorphic 
to $\bigoplus_i N_{\lambda_i}$, where $\lambda_i\in \dQ$.
\end{theorem}
The $N_{\lambda_i}$ are ``isocrystals of slope $\lambda_i$.'' If 
$\lambda=\frac s r$ with $(s,r)=1$ and $r>0$, then as a vector space, 
$N_\lambda=K_0^{\oplus r}$ if $0\leqslant s<r$. The Frobenius is the matrix 
\[
  \varphi_{N_\lambda} = 
  \begin{pmatrix}
    & & & & p \\
    & & & & & \ddots \\
    & & & & & & p \\
    1 \\
    & \ddots \\
    & & 1
  \end{pmatrix}
\]
where there $r$ $p$s and $(r-s)$ 1s. The other cases are obtained by twisting. 

If all the $\lambda_i\in [0,1]$, then these come from formal $p$-divisible groups 
(which are unique up to isogeny). 

\begin{question}
Are there objects that correspond to general isocrystals of slopes $\geqslant 0$? 
\end{question}





\subsection{The universal cover}

Let $G$ be a $p$-divisible group. For convenience, assume $G$ is actually a formal 
group. 

\begin{definition}
The \emph{universal cover} of $G$ is the functor 
\begin{align*}
  \widetilde G &:k\textnormal{-}\mathsf{Alg} \to \mathsf{Set} \\
  \widetilde G(R) &= \varprojlim_{\times p} G(R) 
\end{align*}
\end{definition}
(You may have to take associated fpqc sheaf for this to work.)

\begin{proposition}[Fargues, Fontaine]
Let $d=\dim G$. Then $\widetilde G$ is represented by 
$\spf k\pow{T_1^{1/p^\infty},\dots,T_d^{1/p^\infty}}$. 
\end{proposition}

Consider the ring $k\pow{T_1^{1/p^\infty},\dots,T_n^{1/p^\infty}}$ as an inverse limit 
of its quotients by ideals consisting of power series with degree 
$\geqslant i$. 

\begin{definition}
\begin{enumerate}
  \item A $k$-algebra $R$ is \emph{semiperfect} if $\Phi:R\to R$ is surjective. Then 
    $R^\flat=\varprojlim_\Phi R$ surjects to $R$. Let 
    $I=\ker(R^\flat\twoheadrightarrow R)$. 
  \item A semiperfect ring $R$ is \emph{balanced} if $\Phi(I) = I^p\subset R^\flat$
\end{enumerate}
\end{definition}

\begin{example}
The ring $k[T^{1/p^\infty}/T$ is balanced. One has $\Phi(I)=I^p=(T^p)$. 
\end{example}

\begin{example}
The ring $k[T_1^{1/p^\infty},T^{1/p^\infty}] / (T_1,T_2)$ has 
$\Phi(I)=(T_1^p,T_2^p)$, but $I^p=(T_1^p,T_1^{p-1} T_2,\dots,T_2^p)$. So this 
ring is not balanced. 
\end{example}

Note that $\widetilde G$ is determined by its values on balanced semiperfect 
rings. 

\begin{proposition}[Fontaine]
If $R$ is a semiperfect ring, then there is a universal $p$-adically complete 
PD-thickening $\acris(R)\twoheadrightarrow R$. 
\end{proposition}
\begin{proof}
We briefly recall the construction of $\acris(R)$. It is the $p$-adic completion of 
the PD hull of the kernel of $W(R^\flat)\twoheadrightarrow R$. 
\end{proof}

For example, $\acris(\cO_C/p)$ is the usual ring $\acris$. In general, the ring 
$\acris(R)$ may have lots of $p$-torsion. 

\begin{proposition}
If $R$ is balanced, then $W(R^\flat) \to \acris(R)$ is injective. 
\end{proposition}

This is false in general. 

Recall that $\bcris^+(R)=\acris(R)[\frac 1 p]$. We have a Frobenius morphism 
$\varphi$ acting on both rings. 

\begin{theorem}[Scholze, Weinstein, rem.\ of Lam]
Let $G$ be a $p$-divisible group over $k$ with universal cover $\widetilde G$ and 
isocrystal $(N,\varphi)$. Then for any balanced semiperfect $R$, one has 
\[
  \widetilde G(R) = \hom_\varphi(N,\bcris^+(R)) .
\]
\end{theorem}
\begin{proof}
We will define the map. Recall that 
$\widetilde G(R)=\hom_R(\dQ_p/\dZ_p,G)[\frac 1 p]$. Thus $\widetilde G(R)$ is 
\[
  \hom_{\bcris^+(R),\varphi}(N\otimes_{K_0} \bcris^+(R),\bcris^+(R)) = \hom_\varphi(N,\bcris^+(R)) .
\]
\end{proof}

In some sense, this is the fully-faithfulness of the Dieudonn\'e-module functor 
over $R$. 

Note that the right-hand-side in the theorem makes sense even if $N$ does not come 
from some $G$. 

\begin{definition}
Let $N$ be any $k$-isocrystal of slope $\geqslant 0$. Define 
$\widetilde G_N$ as a functor from balanced semiperfect $k$-algebras of slope 
$>0$. to sets, by 
\[
  R\mapsto \hom_\varphi(N,\bcris^+(R)) .
\]
\end{definition}

\begin{question}
Is $\widetilde G_N$ representable, i.e. $\widetilde G_N=\spf(R_N)$ for some 
inverse limit $R_N$ of balanced semiperfect rings?
\end{question}

The Frobenius $\Phi:R\to R$ induces a bijection $\widetilde G_N(R) \xrightarrow\sim \widetilde G_N(R)$. 
So such a $R_N$ would be the perfection of some balanced semiperfect ring. 

Let's specialize to the first interesting case: $N=N_2$. So 
$N=(K_0,\varphi=p^2)$. In this case we put 
$\widetilde G_2 = \widetilde G_{N_2}$, which is 
$R\mapsto \bcris^+(R)^{\varphi=p^2}$. There is a good candidate for a functor 
representing this ring. We have a map 
$\widetilde G_1\otimes \widetilde G_1 \to \widetilde G_2$, which comes from 
the multiplication map 
\[
  \bcris^+(R)^{\varphi=p}\otimes \bcris^+(R)^{\varphi=p} \to \bcris^+(R)^{\varphi=p^2} .
\]
Recall that $\widetilde G_1=\widetilde{\mu_{p^\infty}}$, so 
$\widetilde G_1\times \widetilde G_1=\spf(k\pow{X^{1/p^\infty},Y^{1/p^\infty}})$. 

Note that $\dQ_p^\times$ acts on $\widetilde G_1$, via 
$X\mapsto (1+X)^\gamma-1$ for $\gamma\in \dQ_p^\times$. To make this precise, 
write $\gamma=p^i\gamma_0$ for $\gamma_0\in \dZ_p^\times$. Then 
\[
  ((1+X)^{p^i})^{\gamma_0}-1 = \sum_{n\geqslant 1} \binom{\gamma_0}{n} X^{p^i-n} .
\]
We get a map 
\[
  \spf\left(k\pow{X^{1/p^\infty},Y^{1/p^\infty}}\right) / (\dZ/2 \times \dQ_p^\times) \to \widetilde G_2 
\]
If $\widetilde G_2$ is representable, the representing object should be 
$R_2^?$, which is the ring of invariants 
\[
  R_2^? := k\pow{X^{1/p^\infty},Y^{1/p^\infty}}^{\dZ/2 \times \dQ_p^\times} .
\]

I suspect that if $\widetilde G_2$ is representable, then the representing object is 
this ring $R_2^?$. 

If $\widetilde G_2$ is represented by $R_2$, then there is a map 
$R_2\to R_2^?$, so $R_2^?$ has to be quite big. 

\begin{question}
Is $R_2^?\ne k$?
\end{question}

If $k\pow{X^{1/p^\infty},Y^{1/p^\infty}}^{\dZ_p^\times}\ne k$, then 
$R_2^?\ne k$. One shows this by summing up over iterates of Frobenius. 
The group $\dZ_p^\times$ is (up to something finite) a topologically cyclic 
group (at least if $p>2$). 

Let's look at the action of $1+p\dZ_p$, which we can regard as the group of 
transformations of the form $f(X)=X+a_1 X^p + a_2 X^{p^2} + \cdots$. Topologically, the latter group is generated by $X + p X^p$. The task 
is to find $F(X,Y)\in k\pow{X^{1/p^\infty},Y^{1/p^\infty}}$ such that for 
$f(X)=X_X^p$, $f^{-1}(X)=X+X^p+X^{p^2} + \cdots$, one has 
\[
  F(X-X^p,Y+Y^p+Y^{p^2}+\cdots ) = F(X,Y) .
\]
Alternatively, we could look for $F=F(X,Y)$ such that 
\[
  F(X+X^p,Y) = F(X,Y+Y^p) .
\]




