\documentclass{article}

\usepackage{msri-style}

\title{Almost ring theory 2: perfectoid rings}
\author{Bhargav Bhatt}
\date{February 17, 2014}

\begin{document}
\maketitle





Here we treat the affine case of the tilting correspondence for perfectoid 
spaces. In general, if $R$ is a ring of characteristic $p$, let 
$\Phi:R\to R$ be the Frobenius $x\mapsto x^p$. 





\section{Tilting for fields}

Let $K$ be a perfectoid field. 

\begin{definition}
Let $K^{\flat\circ} = \varprojlim_\Phi K^\circ/p$. Let 
$K^\flat=\operatorname{Frac}(K^{\flat\circ})$. 
\end{definition}

\begin{theorem}[Fontaine-Wintenberger]
If $K$ is perfectoid, so is $K^\flat$, and tilting induces an equivalence of 
categories $K_\fet\simeq K_\fet^\flat$. 
\end{theorem}

One proves this by starting with $K$, passing to the integral level via 
$K^\circ$, and noting 
that $K^{\flat\circ}\twoheadrightarrow K^{\flat\circ}/\pi \simeq K^\circ/p$ for 
some $\pi$ with $|p|\leqslant |\pi|<1$. Our goal is to generalize this by 
defining categories of perfectoid algebras over $K$ and $K^\flat$, and proving 
that these categories are equivalent via a generalized tilting functor. 

It is not so obvious, but every perfectoid field of characteristic $p>0$ is 
the tilt of a perfectoid field of characteristic zero. 





\section{Tilting more general algebras}

Let $K$ be a perfectoid field, and choose some nonzero $\pi\in\fm$ with 
$|p|\leqslant |\pi| < 1$. 

\begin{definition}
\begin{enumerate}
  \item A Banach $K$-algebra $R$ is \emph{perfectoid} if $R^\circ\subset R$ is 
    open and bounded, and $\Phi:R^\circ/p \to R^\circ/p$ is surjective. Let 
    $K\Perf$ be the category of perfectoid $K$-algebras and continuous 
    morphisms of $K$-algebras. 
  \item A \emph{perfectoid $K^{\circ a}$-algebra} is a flat 
    $K^{\circ a}$-algebra $R$ that is $\pi$-adically complete and Frobenius 
    induces an isomorphism $R/\pi^{1/p} \xrightarrow\sim R/\pi$. Let 
    $K^{\circ a}\Perf$ be the category of perfectoid algebras over 
    $K^{\circ a}$. 
  \item A perfectoid $(K^{\circ a}/\pi)$-algebra $R$ is a flat 
    $(K^{\circ a}/\pi)$-algebra $R$ such that Frobenius induces an isomorphism 
    $R/\pi^{1/p} \xrightarrow \sim R$. Let $(K^{\circ a}/\pi)\Perf$ be the 
    category of perfectoid $(K^{\circ a}/\pi)$-algebras. 
\end{enumerate}
\end{definition}

\begin{example}
Let $R=K\langle T^{1/p^\infty}\rangle$ in the sense of Tate, i.e. 
\[
  R=\left(\bigcup_{n\geqslant 1} K^\circ[T^{1/p^n}]\right)^\wedge[\frac 1 p] 
\]
\end{example}

\begin{example}
Similarly, $A=(K^\circ\langle T^{1/p^\infty}\rangle)^a$ is a perfectoid 
$K^\circ$-algebra. 
\end{example}

\begin{example}
If $K$ has characteristic $p>0$, then perfectoid $K$-algebras are simply 
Banach $K$-algebras $A$ such that $A^\circ$ is open and bounded. 
\end{example}

\begin{theorem}[Tilting equivalence]
There are natural equivalences of categories 
\begin{align*} \tag{a}\label{eq:a}
  K\Perf &\xleftarrow\sim K^{\circ a}\Perf \\ \tag{c}\label{eq:c}
    &\xrightarrow\sim (K^{\circ a}/\pi)\Perf \\ 
    &= (K^{\flat\circ a}/\pi)\Perf \\ \tag{d}\label{eq:d}
    &\xrightarrow\sim K^{\flat\circ a}\Perf \\ \tag{b}\label{eq:b}
    & \xrightarrow\sim K^\flat\Perf
\end{align*}
\end{theorem}

The equivalences \eqref{eq:a} and \eqref{eq:b} use almost mathematics, while 
\eqref{eq:c} and \eqref{eq:d} use some deformation theory. 





\section{Comparing $K\Perf$ and $K^{\circ a}\Perf$} 

\begin{lemma}\label{lem:tech-1}
Let $M$ be a $K^{\circ a}$-module, $N$ a $K^\circ$-module. Then 
\begin{enumerate}
  \item $M$ is flat in over $K^{\circ a}$ if and only if $M_\ast$ is flat over $K^\circ$. 
  \item If $N$ is flat over $K^\circ$, then $N^a$ is flat, and 
    \[
      N_\ast^a = \{x\in N[\frac 1 \pi] : \epsilon x\in N\text{ for all }\epsilon\in\fm\} 
    \]
  \item If $M$ is flat over $K^{\circ a}$, then $M$ is $\pi$-adically complete if 
    and only if $M_\ast$ is $\pi$-adically complete
\end{enumerate}
\end{lemma}

\begin{theorem}
The functors 
\begin{align*}
  K\Perf &\to K^{\circ a}\Perf && R\mapsto (R^\circ)^a \\
  K^{\circ a}\Perf &\to K\Perf && A\mapsto A[\frac 1 \pi] 
\end{align*}
are an equivalence of categories. 
\end{theorem}

The proof of this theorem relies in part on the following proposition. 
\begin{proposition}
If $R$ is perfectoid over $K$, then $R^{\circ a}$ is perfectoid over $K^{\circ a}$. 
\end{proposition}
\begin{proof}
By assumption, $\Phi:R^\circ/\pi \to R^\circ/\pi$ is surjective. We show that 
its kernel is $\pi^{1/p}\cdot R^\circ$. Let 
$x\in R^\circ$ such that $x^p\in \pi\cdot R^\circ$. Then $x^p=\pi\cdot y$ for some 
$y\in R^\circ$. Thus $(x/\pi^{1/p})^p = y\in R^\circ$, so 
$x/\pi^{1/p}\in R^\circ$, whence 
$x\in \pi^{1/p} R^\circ$. 

To see that $R^{\circ a}$ is $\pi$-adically complete, refer to Lemma \ref{lem:tech-1}. 
\end{proof}

\begin{proposition}
Let $A$ be a perfectoid $K^{\circ a}$-algebra. Define $R=A_\ast[\frac 1 \pi]$, with 
the Banach structure making $A_\ast$ open and bounded. Then $A_\ast = R^\circ$ 
and $R$ is perfectoid. 
\end{proposition}
\begin{proof}
First, note that $\Phi:A_\ast/\pi^{1/p} \to A_\ast/\pi$ is injective by Lemma 
\ref{lem:tech-1}. Alternatively, we know that it is an almost isomorphism. So 
for $x\in A_\ast$ with $x^p\in \pi\cdot A_\ast$, we have  
$\epsilon\cdot x\in \pi^{1/p}\cdot A_\ast$ for all $\epsilon\in\fm$. It follows 
that $x\in ((\pi^{1/p}\cdot A_\ast)^a)_\ast = \pi^{1/p}\cdot A_\ast$, whence 
$\ker(\Phi)=0$. 

Next, we assume $x\in R$ has $x^p\in A_\ast$ and show that $x\in A_\ast$. Write 
$y=\pi^{k/p} x\in A_\ast$ for some $k\geqslant 1$. Raise both sides to the $p$ 
power, to obtain $y^p=\pi^k x^p\in \pi A_\ast$. By the injectivity we have already 
proved, we get $y\in \pi^{1/p} A_\ast$. Thus 
$\pi^{(k-1)/p} x\in A_\ast$, so we can apply induction. 

Finally, we show that $\Phi:A_\ast/\pi^{1/p} \to A_\ast/\pi$ is surjective. Because 
the map is already almost surjective, it is enough to show that the composite 
$A_\ast/\pi^{1/p} \to A_\ast/\pi \twoheadrightarrow A_\ast/\fm$ is surjective. 
Choose $x\in A_\ast$. By almost surjectivity, we can write 
$\pi^{1/p}\cdot x \equiv y^p\mod \pi A_\ast$ for some $y$. Let $z=y/\pi^{1/p^2}\in R$. Then 
$z^p = y^p/\pi^{1/p} = x\mod \pi^{1-1/p} A_\ast$, so $z^p\in A_\ast$. We already 
have shown that this implies $z\in A_\ast$, so $z^p=q\mod \pi^{1/p} A_\ast$ implies 
that $x$ has a $p$-th root modulo $\fm$. 
\end{proof}





\section{Some deformation theory}

We'll start with a review of the cotangent complex as it appears in standard 
algebraic geometry. If $A\to B$ is a ring homomorphism, then the cotangent 
complex $L_{B/A}$ is an object in $\mathsf D^{\leqslant 0}(B\Mod)$. If 
$A\to B$ is smooth, then $L_{B/A} = \Omega_{B/A}^1[0]$. Most importantly, 
$L_{B/A}$ controls the deformation theory of $A\to B$ in full generality. 

\begin{example}
Suppose $A=\dF_p$ and $B$ is any perfect $\dF_p$-algebra. It turns out that 
$L_{B/A}=0$. This is easy to prove. By assumption, the Frobenius map 
$\Phi:B\to B$ is an isomorphism, so the induced morphism 
$d\Phi:L_{B/A} \to L_{B/A}$ is an isomorphism. But $d\Phi=0$ because 
$d(y^p) = p y^{p-1} d y = 0$. 
\end{example}

\begin{corollary}
There exists a unique flat $(\dZ/p^n)$-algebra $W_n(B)$ lifting $B$ from 
$\dF_p$ to $\dZ/p^n$. 
\end{corollary}

One typically calls $W_n(B)$ the ring of $p$-typical Witt vectors in $B$ 
of length $n$. 





\section{Comparing $K^{\circ a}\Perf$ and $(K^{\circ a}/\pi)\Perf$}

If $A\to B$ is a map of $K^{\circ a}$-algebras, then \cite{gr03} define 
$L_{B/A}$ as an object of $\mathsf D(B\Mod)$. This $L_{B/A}$ is actually 
constructed as an ``honest complex,'' and satisfies the desired properties. 

\begin{lemma}\label{lem:perf-cot}
Let $A$ be a perfectoid $(K^{\circ a}/\pi)$-algebra. Then 
$L_{A/(K^{\circ a}/\pi)}=0$. 
\end{lemma}
\begin{proof}
Essentially, this comes down to the fact that the relative Frobenius is an 
isomorphism. Consider the following commutative diagram:
\[\xymatrix{
  A 
    & A/\pi^{1/p} \ar[l]_-\sim
    & A \ar[l] \\
  K^{\circ a}/\pi \ar[u] 
    & K^{\circ a}/\pi^{1/p} \ar[l]_-\sim \ar[u] 
    & K^{\circ a}/\pi \ar[u] \ar@{->>}[l] 
}\]
Tne previous argument completes the proof. 
\end{proof}

Lemma \ref{lem:perf-cot} implies that 
$(K^{\circ a}/\pi)\Perf \simeq (K^{\circ a}/\pi^n)\Perf \simeq K^{\circ }\Perf$. 
Explicitly, start with $A\in (K^{\flat \circ a}/t)\Perf$. Then 
$A^\flat = \varprojlim_\Phi A$ is a perfectoid $K^{\flat\circ a}$-algebra. 





\section{Tilting \'etale covers}

Let $A$ be a perfectoid algebra over $K$. Our goal is to prove that there are  
natural equivalences 
\begin{align*} \tag{a}
  A_\fet &\xleftarrow\sim A^{\circ a}_\fet \\ \tag{b}
  &\xrightarrow\sim (A^{\circ a}/\pi)_\fet \\
  &= (A^{\flat\circ a}/t)_\fet \\ \tag{d}
  &\xleftarrow\sim (A^{\flat\circ a})_\fet \\ \tag{c}
  &\xrightarrow\sim A^\flat_\fet
\end{align*}
We have seen that \eqref{eq:b} and \eqref{eq:d} follow from deformation theory, 
and \eqref{eq:c} follows from the last example. 





\bibliographystyle{alpha}
\bibliography{msri-sources}


\end{document}
