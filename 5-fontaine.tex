\documentclass{article}

\usepackage{msri-style}

\title{Introduction to $p$-adic comparison theorems}
\author{Jean-Marc Fontaine}
\date{February 18, 2014}

\begin{document}
\maketitle





Unfortunately, titles of talks were not given by the speakers. So this talk 
will contain nothing about $p$-adic comparison theorems. A better title would 
be ``constructing universal rings.'' 





\section{Rings of Witt vectors}

Fix a prime number $p>0$. Cuntz and Denninger discovered a very beautiful 
way of constructing the ring of Witt vectors $W(R)$, for $W$ a perfect 
$\dF_p$-algebra. It involves some ``naive multiplicative deformation theory.'' 
For more details, see their preprint \cite{cd13}. 

Let $\Lambda$ be a commutative ring, $R$ a $\Lambda$-algebra $n>0$ an integer. 
Recall that a \emph{$\Lambda$-infinitesimal thickening} of order 
$\leqslant n$ of $R$ is a pair $(A,I)$, where $A$ is a $\Lambda$-algebra, 
$I\subset A$ is an ideal, together with an isomorphism $A/I\to R$ of 
$\Lambda$-algebras, such that $I^{n+1}=0$. It is natural to ask whether the 
category of such thickenings has an initial object. The answer is no in 
general, so we consider the category of triples $(A,I,\sigma)$, where 
$(A,I)$ is a $\Lambda$-thickening of $R$ and $\sigma:R\to A$ is a 
multiplicative section of $A\twoheadrightarrow R$. The category of such triples 
has an initial object which can be constructed explicitly. 

Start with the monoid ring $\Lambda[R]$ consisting of formal sums 
$\sum_{x\in R} \lambda_x [x]$, with all but finitely may $\lambda_x=0$. 
There is a ring homomorphism (called the augmentation map) 
$\varepsilon:\Lambda[R^\times] \to R$, sending $\sum \lambda_x [x]$ to 
$\sum \lambda_x x$. Put 
$U_{n,\Lambda}(R) = \Lambda[R^\times]/(\ker\varepsilon)^{n+1}$. Clearly 
$(U_{n,\Lambda}(R),\ker\varepsilon)$ is a $\Lambda$-infinitesimal thickening of 
$R$ of order $\leqslant n$. There is an obvious multiplicative section 
$\sigma:R\to U_{n,\Lambda}(R)$ given by $r\mapsto [r]$. It is easy to check 
that $(U_{n,\Lambda}(R),\ker\varepsilon,\sigma)$ is the universal 
``$\Lambda$-infinitesimal thickening of order $\leqslant n$ with section'' of 
$R$. 

If $R$ is a perfect $\dF_p$-algebra, then $U_{n,\dZ}(R)=U_{n,\dZ_p}(R)$ is 
$W_n(R)$, the ring of length-$n$ Witt vectors of $R$, so 
$W(R) = \varprojlim_n U_{n,\dZ}(R)$. If $R$ is a perfect $\dF_p$-algebra, then 
given any $\Lambda$-infinitesimal thickening $(A,I)$ of $R$, we can form a 
multiplicative section $\sigma:R\to A$ by 
$x\mapsto (\widetilde{x^{p^{-m}}})^{p^m}$, for $\widetilde{x^{p^{-m}}}$ a lifting 
of $x$ to $A$, and $m\gg 0$. 





\section{Tilting revisited}

Let's start by defining the class of rings we'll be working with. 

\begin{definition}\label{def:banach-ring}
A \emph{Banach ring} $A$ is a topological ring containing a pseudo-uniformizer 
(topologically nilpotent invertible element) $\pi$, and an open subring 
$A_0\ni \pi$, such that 
$A = A_0[\frac 1 \pi]$, and such that $A_0 \to \varprojlim A_0/\pi^n$ is an 
isomorphism of topological rings.  
\end{definition}

Note that a Banach ring is simply a complete Tate algebra (in the sense of 
Huber). Recall that if $A$ is a Banach ring, then a set $S\subset A$ is called 
\emph{bounded} if for all neighborhoods $U$ of $0$, there exists 
$n\geqslant 1$ so that $\pi^n S\subset U$. An element $a\in A$ is 
\emph{power-bounded} if the set $\{a^n:n\geqslant 0\}$ is bounded, and we 
write $A^\circ\subset A$ for the subring of power-bounded elements. 

\begin{definition}
A \emph{spectral ring} is a Banach ring $A$ such that $A^\circ$ is bounded. 
\end{definition}

If $A$ is a Banach ring, then $A$ is spectral if and only if there exists a 
power-multiplicative norm inducing the topology. Recall that a 
\emph{power-multiplicative norm} on $A$ is a map 
$|\cdot|:A\to \dR_{\geqslant 0}$ such that $|a b|\leqslant |a|\cdot |b|$, 
$|1|=1$, and $|a^n|=|a|^n$ for $n\geqslant 0$. If $A$ is complete with 
respect to $|\cdot|$, one has $A^\circ = \{a\in A:|a|\leqslant 1\}$. If $A$ is 
a Banach ring, we can construct a power-multiplicative norm directly. Choose 
$\pi\in A$ as in Definition \ref{def:banach-ring}. Also choose $\rho\in \dR$ 
with $0<\rho<1$. For $a\in A$, put 
\[
  v_\pi(a) = \sup\left\{\frac r s :  r\in \dZ,s>0\text{ and }\frac{a^s}{\pi^r}\in A^\circ\right\} .
\]
One has $v_\pi(a)\in \dR_{\geqslant 0}$, so it makes sense to define 
$|a| = \rho^{v_\pi(a)}$. 

\begin{definition}
A \emph{perfectoid ring} is a spectral ring $A$, such that there exists a 
power-bounded unit $\pi\in A^\circ$ such that $p\in \pi^p A^\circ$ and the 
Frobenius map $A^\circ/\pi^p A^\circ$ is surjective. 
\end{definition}

So a perfectoid field is a perfectoid ring which is a field, and which admits a multiplicative norm. 

If $A$ is a spectral ring of positive characteristic, then $A$ is perfectoid 
if and only if it is perfect. Moreover, if $\pi\in A$ is any 
pseudo-uniformizer, then $A$ is naturally an algebra over the perfectoid field 
$\dF_p\lau{\pi^{1/p^\infty}} = \widehat{\dF_p\lau\pi (\pi^{1/p^\infty})}$. 

If the characteristic of $A$ is not $p$, then $A$ is a Banach $p$-adic algebra, 
so $\dQ_p\hookrightarrow A$, but there is no canonical perfectoid subfield of 
$A$. 

Let $A$ be a perfectoid ring of characteristic zero. One can define a new ring 
$A^\flat$, called the \emph{tilt} of $A$, in two different but equivalent ways. 
First, choose a pseudo-uniformizer $\pi\in A^\circ$ with $p\in \pi A^\circ$, 
and put 
\[
  A^{\flat\circ} = \varprojlim_\Phi A^\circ/\pi
\]
where $\Phi:x\mapsto x^p$ is Frobenius. Define 
$A^\flat=A^{\flat\circ}[\frac 1 \pi]$. Alternatively, we could have defined 
$A^\flat$ as a set to be  
\[
  A^\flat = \varprojlim_{x\mapsto x^p} A = \left\{(x^{(n)})_{n\geqslant 0}:(x^{(n+1)})^p = x^{(n)}\right\} .
\]
This has an obvious multiplication operation, but the addition is not obvious. 
We define it directly by 
\[
  (x+y)^{(n)} = \lim_{m\to \infty} (x^{(n+m)} + y^{(n+m)})^{p^m} .
\]
Define $(-)^\sharp:A^\flat \to A$ given by 
$(x^{(n)})_{n\geqslant 0} \mapsto x^{(0)}$. 

If $A$ is a perfectoid ring of characteristic zero, put $R=A^\flat$; this is 
a perfectoid ring of characteristic $p$. There is a natural surjection 
$\theta_A:W(R^\circ) \twoheadrightarrow A^\circ$, defined by 
\[
  \theta_A\left(\sum_{i=0}^\infty p^i [a_i]\right) = \sum_{i=0}^\infty p_i a_i^{(0)} = \sum_{i=0}^\infty p^i a_i^\sharp
\]
where each $a_i=(a_i^{(n)})_{n\geqslant 0}$. The kernel of 
$\theta_A$ is a primitive ideal of degree one. In other words, it can be 
generated by an element of the form $[\pi]+p\eta$, where $\pi$ is a 
pseudo-uniformizer of $R$, and $\eta\in W(R^\circ)^\times$. 

The operation $A\mapsto (A^\flat, \ker\theta_A)$ is a functor from the category 
of perfectoid rings of characteristic zero to the category of ``perfectoid 
pairs.'' Here a perfectoid pair is a pair $(R,I)$, where $R$ is a perfectoid 
ring of characteristic $p>0$, and $I\subset W(R^\circ)$ is a primitive ideal 
of degree one. 

\begin{theorem}
This functor is an equivalence of categories.
\end{theorem}
\begin{proof}
We construct an inverse. Given a perfectoid pair $(R,I)$, define 
$A=(R,I)^\sharp$ by $A^\circ =W(R^\circ)/I$ and $A=A^\circ[\frac 1 p]$. 
Checking that $(-)^\sharp$ is an inverse to 
$A\mapsto (A^\flat, \ker\theta_A)$ is a straightforward exercise. 
Alternatively, see Kedlaya's note \cite{ke13}. 
\end{proof}





\section{Tilting and analytic extensions}

We know that for any perfectoid field $K$ of characteristic zero, the tilt 
$K^\flat$ is a perfectoid field of characteristic $p$. On the other hand, 
suppose we start out with a perfectoid field $F$ of characteristic $p$. 
Then to any primitive ideal $I\subset W(F^\circ)$, the above construction 
yields a perfectoid field $(F,I)^\sharp$ of characteristic zero, whose tilt is 
$F$. Thus every perfectoid field of characteristic $p$ is the tilt of a 
(explicitly constructed) perfectoid field of characteristic zero. 

Suppose $K_0$ is a finite extension of $\dQ_p$. Let $L$ be an ``analytically 
profinite extension of $K_0$.'' In other words, $L$ could be an algebraic extension 
of $K_0$ which is infinitely ramified and such that the Galois group of the Galois 
closure of $L/K_0$ is a $p$-adic Lie group. Clearly any analytically profinite 
extension of $K_0$ gives a $p$-adic representation of $G_{K_0}$. 

Let $K$ be the completion of $L$. To $K$ we associate a ``norm field,'' which is 
isomorphic to $k_L\lau t$, where $k_L$ is the residue field of $L$. The 
``Fontaine-Wintenberger theorem'' is that $L$ and $k_L$ have the same 
\'etale sites. Moreover, $K$ is perfectoid and 
$K^\flat$ is the perfectoid completion of $k_L\lau t$. 

If we took $K=\dC_p$, then we get an algebraically closed perfectoid field 
$F$ of characteristic $p$. It is unknown if all the untiltings of this field 
are isomorphic. 

Let $K$ be a fixed perfectoid field of characteristic zero. Then $K^\flat$ is a 
perfectoid field of characteirstic $p$. We know that the kernel of  
$\theta_K:W(K^{\flat\circ}) \to K^\circ$ is principal; let $\xi$ be any 
generator. If $A$ is any perfectoid $K$-algebra, then 
$(A^\flat,(\xi))$ is a perfectoid pair. 





\section{Period rings}

Let $k$ be a perfect field of characteristic $p$, and fix an ultrametric field 
$E$ with residue field $k$. For any perfect $k$-algebra $R$ of characteristic 
$p$, we will construct a ``bounded period ring'' $\bB_E^\text{b}(R)$. Start by 
defining $W_{E^\circ}(R)$ to be the completed tensor product 
\[
  E^\circ\widehat\otimes_{W(k)} W(R) = \varprojlim (E^\circ/\pi^n)\otimes_{W(k)} W(R) ,
\]
where $\pi\in E^\circ$ is any pseudo-uniformizer. 

The surjection $W_{E^\circ}(R) \twoheadrightarrow R$ has a multiplicative 
section $[-]:R\to W_{E^\circ}(R)$, given by $x\mapsto 1\otimes [x]$. If $E$ 
has characteristic $p$, this section is additive and we can view $R$ as a 
subring of $W_{E^\circ}(R) = E^\circ\widehat\otimes_k R$. 

Suppose $R$ is a perfectoid $k$-algebra. (Since $k$ is not perfectoid, we simply 
assume $k\subset R^\circ$.) Let $\varpi$ be a pseudo-uniformizer of $R$, and 
define 
\[
  \bB_E^\text{b}(R) = W_{E^\circ}(R^\circ)\left[\frac 1 \pi, \frac{1}{[\varpi]}\right] .
\]
If $E$ is discretely valued, we can choose $\pi$ to be a uniformizer of $E$, 
and  
\[
  \bB_E^\text{b}(R) = \left\{\sum_{i\gg -\infty} [a_i] \pi^i: a_i\in R\text{ bounded}\right\}
\]
In general, given $(a_n)_{n\geqslant 0}$, with $\{a_n\}\subset R$ bounded, and 
$(\nu_n)_{n\geqslant 0}$, with $\nu_n\in E$ converging to zero, the infinite sum 
$\sum_{n=0}^\infty [a_n] \nu_n \in \mathbf B_E^b(R)$, but this ``decimal expansion'' 
is not unique. 

Choose an absolute value on $E$, and a power-multiplicative norm on $R$. Up to 
equivalence, these are unique. 

\begin{proposition}
For all $\rho\in \dR$, $0<\rho<1$, there exists a unique power multiplicative norm 
$|\cdot |_\rho$ on $\bB_E^\textnormal{b}(R)$ such that 
$|[a]|_\rho = |a|$ for all $a\in R$, and such that $|\pi|=\rho$. 
\end{proposition}

Choose a non-empty closed interval $I=[\rho_1,\rho_2]$, with 
$0<\rho_1\leqslant \rho_2<1$. Define 
\[
  |\alpha|_I = \sup\{|\alpha|_{\rho_1},|\alpha|_{\rho_2}\} = \sup_{\rho_1\leqslant \rho\leqslant \rho_2} |\alpha|_\rho .
\]
Let $\bB_{E/I}(R)$ be the completion of $\bB_E^\text{b}(R)$ with respect to 
the norm $|\cdot |_I$. This is a spectral $E$-algebra. If $E$ is a perfectoid 
field, then $\bB_{E/I}(R)$ is a perfectoid $E$-algebra. The tilt of 
$\bB_{E/I}(R)$ is $\bB_{E^\flat/I}(R)$. 

Now we can define some period rings. For any perfectoid $\dQ_p$-algebra $R$, 
we can define $\mathbf B_\text{cris}^+(A)$ and $\mathbf B_\text{dR}^+(A)$. 
The ``morally correct'' definition of $\bdr^+$ is 
\[
  \bdr^+(A) = \varprojlim_{n\geqslant 0} \bB_n(A).
\]
Here the inverse limit is taken in the category of Banach $\dQ_p$-algebras, and 
$\bB_n(A)$ is the universal divided-power thickening of $A$ of order 
$\leqslant n+1$. For a more explicit definition, start with the homomorphism 
$\theta:W(A^{\flat\circ})[\frac 1 p] \twoheadrightarrow A$, whose kernel is 
a principal ideal $(\xi)$. It turns out that 
$\bB_n(A) = W(A^{\flat\circ})[\frac 1 p]/\xi^n$, so 
\[
  \bdr^+(A) = \varprojlim W(A^{\flat\circ})\left[\frac 1 p\right]/(\ker\theta)^n .
\]
This is a discrete valuation ring with uniformizer $\xi$, so we can define 
$\bdr(A)$ to be the field of fractions of $\bdr^+(A)$, i.e. 
$\bdr(A)=\bdr^+(A)[\frac 1 \xi]$. 

To define $\acris(A)$, start by defining $\acris^\varphi(A)$ to be the 
sub-$W(A^{\flat\circ})$-algebra of $W(A^{\flat\circ})$ generated by 
$\{\xi^n/n:n\geqslant 1\}$. Let $\acris(A)$ be the $p$-adic completion of 
$\acris^\varphi(A)$, and define $\bcris^+(A) = \acris(A)[\frac 1 p]$.





\bibliographystyle{alpha}
\bibliography{msri-sources}

\end{document}
