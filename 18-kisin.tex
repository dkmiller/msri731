% !TEX root = msri731.tex

\section{Shimura varieties and perfectoid spaces 3}
\thanksauthor{Mark Kisin (Feb.\ 21)}





The point is to use the result of the last lecture, along with the Hodge-Tate period 
map, to compute the completed cohomology and its' Hecke action. 

Let $(G,X)\subset (\operatorname{GSp},S^\pm)$ be a Shimura datum of Hodge type. 
Let $K^p\subset G(\dA_f^p)$ be a level, and let $K_p\subset G(\dQ_p)$. Write 
$Y_{K_p K^p}$ for the corresponding Shimura variety, and 
$X_{K_p K^p}$ for a minimal compactification. We think of everything as living over 
$\dC_p$. 





\subsection{A theorem on Hecke actions}

Recall that we defined completed cohomology as 
\[
  \widetilde \h_{c,K^p}^i(\dZ/p^n) = \varinjlim_{K_p \to 0} \h_c^i(Y_{K_p K^p}, \dZ/p^n) .
\]
Let $\dT = \dZ_p[G^p\backslash G(\dA_f^p) / K^p]$ be the corresponding Hecke algebra. 
This acts on $\widetilde \h^i_{c,K^p}(\dZ/p^n)$. 

There is a ``standard'' line bundle $\omega$ on $X_{K_p K^p}$, coming from 
$\Omega^g$ of the universal family of abelian varieties. Let $I$ be the ideal sheaf 
of $X\smallsetminus Y$. The algebra $\dT$ acts on a space of ``modular forms'' 
$\h^0(X_{K_p K^p},\omega^k\otimes I)$. 

\begin{theorem}
Let $\dT_\textnormal{cl}$ be the minimal quotient of $\dT$ such that the action of 
$\dT$ on $\h^0(X_{K_p K^p},\omega^{\otimes k}\otimes I)$ factors through 
$\dT_\textnormal{cl}$ for all $k$, $K_p$. 
Then the action of $\dT$ on $\widetilde\h_{c,K^p}^i(\dZ/p^n)$ factors through 
$\dT_\textnormal{cl}$. 
\end{theorem}

Let $X=\varprojlim_{K_p} X_{K_p K^p}$. This as denoted by 
$X_{G,\Gamma(p^\infty)}$ in the last talk, and we know it is a perfectoid space. 
Let $I^+=\sO^+\cap I$. 

\begin{proposition}
There is an almost isomorphism 
\[
  \widetilde \h_{c,K^p}^i(\dZ/p^n) \otimes \cO_{\dC_p}/p^2 \simeq_a \h^i(X,I^+/p^n I^+) .
\]
\end{proposition}

Write $j:Y\hookrightarrow X$ for the inclusion. We know that 
$\h_{c,K^p}$ comes from $j_! \dZ/p^n$. In other words, $\h_c=\h_{K^p}(j_! \dZ/p^n)$. 
In this case, one has ``one the nose'' that 
$I^+/p^n = j_! \sO^+/p^n$. Near the boundary, one has $I^+ \to \sO^+ \to \sO_\partial^+$. 
To prove the theorem, we need to compute $\h^i(X,I^+/p^n)$. For this we use the 
Hodge-Tate period map $\pi_\text{HT}$, coupled with the fact that $X$ is 
perfectoid. 

We have a diagram 
\[\xymatrix{
  & X_{\operatorname{GSp}} \ar[dr]^-{\pi_\text{HT}} \\
  X \ar[rr]^-{\pi_\text{HT}} \ar[ur] 
    & & F \ar@{^{(}->}[r] 
    & \dP^{\binom{2 g}{g}-1} 
}\]
where $D\subset \dZ_p^{2 g}\otimes K$ is mapped to 
$\bigwedge^g D\subset \bigwedge^g(K^{2 g})$. The pullback of $\sO(1)$ is $\omega$. 
The embedding from $F$ into projective space is via Pl\"uker coordinates or something 
similar? 

For $J\subset \{1,\dots,2 g\}$ with $|J|=g$, let $F_J\subset F$ be the locus where 
$|S_J|\geqslant |S_{J'}|$ for all $J'$. For example, 
\[
  \pi_\text{HT}^{-1}(F_{\{g+1,\dots,2 g\}}(\dQ_p)) = \overline{X_{\Gamma(p^\infty)}(0)_a} .
\]

\begin{lemma}
All $\pi_\textnormal{HT}^{-1}(F_J)$ are affinoid perfectoid. 
\end{lemma}
\begin{proof}
It is enough to consider $J=\{g+1,\dots,2 g\}$. Then $\gamma$ is a diagonal matrix with 
$g$ $p$-s, and then $1$s along the diagonal. Consider $\gamma^n\cdot F_J\subset F_J$. 
These are rational subsets. If $n\gg 0$, then 
$\pi_\text{HT}^{-1}(\gamma^n\cdot F_J)\subset X_{\Gamma(p^\infty)}(\varepsilon)_a$, 
which is an affinoid. 

[\ldots not enough on board, did not follow\ldots]

By the lemma, we cam compute $\h^i(X,I^+/p^n)$ using 
$\{\pi_\text{HT}^{-1}(F_J)\}_J$. Order these open subsets as 
$V_1,\dots,V_N$, where $N=\binom{2 g}{g}$. Let $J_2\subset \{1,\dots,N\}$, and put 
$V_{J_2} = \bigcap_{i\in J_2} V_i$. The $V_i$ are affinoid perfectoid. Then 
$\h^i(V_{J_2},I^+/p^n) =_a 0$ for all $i>0$. 

It is enough to show that the action of $\dT$ on $\h^0(V_{J_2},I^+/p^n)$ factors 
through $\dT_\text{cl}$. In fact, the $V_{J_2}$ come from $V_{J_2,K_p}\subset X_{K_p K^p}$ 
for some $K_p$ small enough. The $V_{J_2K_p}$ are affinoids, and bounded functions 
give a ``natural'' integral model $V_{J_2,K_p}^\circ$. One can ``glue these'' into 
a formal scheme $X_{K_p K^p}^\circ$, equipped with an ample line bundle 
$\omega^\text{int}$ extending $\omega$. 

Finally, 
\begin{align*}
  \h^0(V_{J_2}, I^+/p^n) 
    &= \varinjlim_{K_p} \h^0(V_{J_2,K_p}^\circ, I^+/p) \\
    &= \varinjlim_{K_p} \varinjlim_{S_{J_2}} \h^0(X_{K_p K^p}^\circ, (\omega^\text{int})^{\otimes k |J_2|}\otimes \mathscr J/p^n)
\end{align*}
where $\mathscr J$ is the ideal shaef of the boundary on $X_{K_p K^p}^\circ$. 
\end{proof}

[\ldots to vague to follow + battery ran out\ldots ]








