\documentclass{article}

\usepackage{msri-style}

\title{Adic spaces 1}
\author{Eugene Hellman}
\date{February 17, 2014}

\begin{document}
\maketitle





The main goal is to introduce the category of adic spaces over a 
non-archimedean base field, which generalizes rigid geometry. We want this 
framework to be able to handle more general affinoid rings (no finiteness 
conditions). The structure sheaf should be defined on all opens, not just 
with respect to a Grothendieck topology. 





\section{Affinoid algebras}

Fix a non-archimedean base field $k$, i.e. $k$ is a complete topological field, 
with the topology defined by a nontrivial rank-one valuation 
$|\cdot |:k^\times \to \dR_{\geqslant 0}$. Main examples include $\dQ_p$, 
$\dF_q\lau t$, and any perfectoid field. 

\begin{definition}
An \emph{f-adic} ring is a topological ring $R$ such that $R$ contains an open 
subring $R_0$ (called the \emph{ring of definition}) such that the topology 
on $R_0$ is adic with respect to a finitely generated ideal of definition. 
\end{definition}

\begin{definition}
An f-adic ring $R$ is called a \emph{Tate ring} if there exists a topologically 
nilpotent unit in $R_0$. 
\end{definition}

If $R$ is a topological $k$-algebra, then $R$ is a Tate ring if the sets 
$\{a R_0:a\in k^\times\}$ form a basis for the topology on $R$. 

\begin{example}
Consider the ring 
$k\langle T_1,\dots,T_n\rangle = k^\circ[T_1,\dots,T_n]^\wedge[\frac 1 \pi]$, 
where $(-)^\wedge$ denotes $\pi$-adic completion for some $\pi\in k^\times$ with 
$|\pi|<1$. 
\end{example}

An element $a\in R$ is called \emph{power-bounded} if the set 
$\{a^n:n\geqslant 0\}$ is a bounded subset of $R$. In this context, a set 
$S\subset R$ is \emph{bounded} if for any open neighborhood $U$ of $0$, there 
exists $a\in k^\times$ such that $a U\supset S$. Write $R^\circ$ for the 
subring of $R$ consisting of power-bounded elements. 

\begin{definition}
An \emph{affinoid algebra} is a pair $(A,A^+)$ with $A$ an f-adic ring and 
$A^+\subset A^\circ$ an open integrally closed subring. 
\end{definition}

We say that an affinoid algebra $(A,A^+)$ is of finite type over $k$ if 
$A$ is a quotient of $k\langle T_1,\dots,T_n\rangle$ for some $n$, and 
$A^+=A^\circ$. From now on, let $A$ denote either an f-adic ring or a Tate 
ring. 





\section{Valuations}

\begin{definition}
A \emph{valuation} on a ring $A$ is a map $v:A\to \Gamma\cup\{0\}$, where 
$\Gamma$ is a totally ordered abelian group (written multiplicatively), such 
that 
\begin{enumerate}
  \item $v(0)=0$ and $v(1)=1$ 
  \item $v(a b) = v(a) v(b)$ 
  \item $v(a+b)\leqslant \max\{v(a),v(b)\}$
\end{enumerate}
\end{definition}
Here as is typical, we put $\gamma>0$ for all $\gamma\in \Gamma$, and define 
$\gamma\cdot 0 = 0\cdot \gamma = 0$. 

If $v$ is a valuation on $A$, let $\supp(v) = \{x\in A:v(x)=0\}$; this is an 
ideal in $A$. Let $\Gamma_v$ be the subgroup of $\Gamma$ generated by 
$\{v(x):x\in A,v(x)\ne 0\}$. We say that two valuations $v,v'$ are 
\emph{equivalent} if for all $a,b\in A$, one has 
$v(a)\leqslant v(b)\Leftrightarrow v'(a)\leqslant v'(b)$. Two valuations 
$v,v'$ are equivalent in this sense if and only if $\supp(v)=\supp(v')$ and 
the associated valuation rings in $\operatorname{Frac}(A/\supp(v))$ are the 
same. 

Suppose $v$ is a valuation on a topological ring $A$. We call $v$ 
\emph{continuous} if for all $\gamma\in \Gamma_v$, there exists an open 
neighborhood $U$ of zero in $A$ such that $v(x)<\gamma$ for all $x\in U$. 

\begin{definition}
Let $A$ be an f-adic ring. Define $\spv(A)$ to be the set of equivalence 
classes of valuations on $A$. 
\end{definition}

\begin{definition}
If $A$ is an f-adic ring, $\cont(A)$ is the set of equivalences of continuous 
valuations on $A$. 
\end{definition}

\begin{definition}
Let $(A,A^+)$ be an affinoid ring. Define $\spa(A,A^+)$ to be the set of 
continuous valuations $v$ on $A$ such that for all $a\in A^+$, 
$v(a)\leqslant 1$. 
\end{definition}

We equip all the spaces with a topology generated by the sets 
\[
  U_{f,g} = \{v : v(f)\leqslant v(g)\ne 0\}
\]
for $f,g\in A$. 

\begin{theorem}
The spaces $\spv(A)$, $\cont(A)$ and $\spa(A,A^+)$ are spectral spaces. 
\end{theorem}





\section{Adic spectra}

Let $(A,A^+)$ be an affinoid ring. For $x\in \spa(A,A^+)$ corresponding to a 
valuation $v$, and $f\in A$, put $|f(x)| = v(f)$. 

If $A$ is a Tate $k$-algebra, then the rational subsets 
\[
  U\left(\frac{f_1,\dots,f_n}{g}\right) = \{x\in \spa(A,A^+):|f_i(x)\leqslant |g(x)|\text{ for all }i\}
\]
for $f_1,\dots,f_n\in A$ with $(f_1,\dots,f_n) = A$, form a basis for the 
topology on $\spa(A,A^+)$. 

\begin{example}
Let $A$ be a Tate $k$-algebra of finite type. If $\fm\subset A$ is a maximal 
ideal, then we get a valuation on $A$ by composing $A\twoheadrightarrow A/\fm$ 
with the unique extension of $|\cdot |_k$ to $A/\fm$. This gives a canonical 
inclusion 
\[
  \operatorname{Sp}(A)=\{\fm\in\spec(A):\fm\text{ maximal}\}\hookrightarrow \spa(A,A^\circ) .
\]
If $A$ is a finite-type Tate $k$-algebra, then we have a corresponding statement 
for the corresponding rigid analytic space associated to $A$. The topology on 
$\spa(A,A^\circ)$ induces the standard Grothendieck topology on the standard 
rigid-analytic space. One has bijections  
\begin{align*}
  \{\text{q-c opens in }\spa(A,A^\circ)\} 
    &\leftrightarrow \{\text{q-c opens in }\operatorname{Sp}(A)\} \\
  \{\text{coverings by q-c opens}\} 
    &\leftrightarrow\{\text{admissible covers by q-c admissible opens}\}
\end{align*}
\end{example}

We'd like to construct a structure sheaf on $X=\spa(A,A^+)$. For simplicity's 
sake, we assume $(A,A^+)$ is Tate. Let 
$U=U\left(\frac{f_1,\dots,f_n}{g}\right)\subset X$ be a standard rational 
subset; we need to define $\sO_X(U)$. Let 
$A\langle \frac{f_1}{g},\dots,\frac{f_n}{g}\rangle$ be the completion of 
$A[\frac{f_1}{g},\dots,\frac{f_n}{g}]$ with respect to the topology generated 
by opens of the form $a\cdot A_0[\frac{f_1}{g},\dots,\frac{f_n}{g}]$, for 
$a\in k^\times$. (Here $A_0\subset A$ is any ring of definition.) 
Let $A\langle \frac{f_1}{g},\dots,\frac{f_n}{g}\rangle^+$ be the completion of 
the integral closure of $A^+[\frac{f_1}{g},\dots,\frac{f_n}{g}]$ with respect 
to the same topology. 

\begin{proposition}
The canonical map $\spa(A\langle\frac{f_i}{g}\rangle,A\langle \frac{f_i}{g}\rangle^+) \to (\spa A,A^+)$ induced by $\varphi:(A,A^+) \to (A\langle \frac{f_i}{g}\rangle, A\langle \frac{f_i}{g}\rangle^+)$ factors over $U$. Moreover, $\varphi$ is universal for maps 
$(A,A^+) \to (B,B^+)$ with $B$ complete such that $\spa(B,B^+) \to \spa(A,A^+)$ factors over $U$. 
\end{proposition}

For $U$ as above, we define 
$\sO_X(U)=A\langle \frac{f_1}{g},\dots,\frac{f_n}{g}\rangle$ and 
$\sO_X^+(U) = A\langle \frac{f_1}{g},\dots,\frac{f_n}{g}\rangle^+$. For general 
$U$, we put 
\[
  \sO_X(U) = \varprojlim_{\substack{W\subset U \\ U\text{ rational}}} \sO_X(W) .
\]
It is not known whether $\sO_X$ as defined is a sheaf in general. This is 
easily circumvented by the following definition. 

\begin{definition}
Let $(A,A^+)$ be as above. Then $\spa(A,A^+)$ is called an \emph{affinoid adic 
space} with sheaf $\sO_X$ if the presheaf $\sO_X$ is a sheaf. 
\end{definition}

\begin{theorem}
If $A$ is strongly noetherian (i.e.\ $A\langle T_1,\dots,T_n\rangle$ is 
noetherian for all $n\geqslant 0$) then $\sO_X$ is a sheaf on $X=\spa(A,A^+)$. 
\end{theorem}

For example, Tate algebras of finite type over any non-archimedean field $k$ 
are strongly noetherian. 





\section{Glueing of affinoid adic spaces}

One glues adic spaces in the category $\cV$ consisting of triples 
$(X,\sO_X,\{v_x:x\in X\})$, where $(X,\sO_X)$ is a locally ringed space, 
$\sO_X$ is a sheaf of complete topological rings on $X$, and for each 
$x\in X$, $v_x$ is a valuation on $\sO_{X,x}$. Morphisms in $\cV$ are required 
to respect all the data. 

If $(A,A^+)$ is an affinoid ring such that $\spa(A,A^+)$ is an adic space, 
then for $x\in X=\spa(A,A^+)$, there is a canonical extension of the valuation 
$v_x$ to $\sO_{X,x}$. 

It is necessary to keep track of the valuations $\{v_x:x\in X\}$ to be able to 
recover $\sO_X^+$. The corresponding category of triples $(X,\sO_X,\sO_X^+)$ 
is equivalent to $\cV$. 

\begin{definition}
An \emph{adic space} is an object of $\cV$ that is locally isomorphic to an 
affinoid adic space. 
\end{definition}





\end{document}
