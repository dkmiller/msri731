% !TEX root = msri731.tex

\section{Adic spaces 3}
\thanksauthor{Peter Scholze (Feb.\ 18)}


\subsection{\'Etale topology for perfectoid spaces}

Note that perfectoid rings (i.e. spectral rings, in Fontaine's vocabulary) are always 
reduced. If $x\in R$ is nilpotent, then any $\varpi^{-n} x\in R^\circ$, so 
$K\cdot x\subset R^\circ$, but this is impossible if $R^\circ$ is bounded. So we 
can't use an infinitesimal lifting criterion to define ``\'etaleness.'' However, 
we have the following proposition. 

\begin{proposition}
For adic spaces of finite type over $K$, open embeddings and finite \'etale covers 
generate the \'etale site. 
\end{proposition}

Note that this is not true for schemes. One defines finite \'etale covers using 
Bhargav's definition for global sections. 

\begin{definition}
A map $f:Y\to X$ of perfectoid spaces is \emph{etale} if it is locally of the form 
$Y\hookrightarrow Z\twoheadrightarrow W\hookrightarrow X$, where 
$Y\hookrightarrow Z$ is an open embedding, $Z\twoheadrightarrow W$ is finite \'etale, 
and $W\hookrightarrow X$ is open. 
\end{definition}

One can check that this notion of \'etaleness satisfies the expected properties (e.g. 
closure under composition). Thus we can define the \'etale site $X_\et$. Moreover, 
if a map becomes \'etale after a change of the ground field, it was \'etale to 
begin with. 

\begin{proposition}
If $X$ is a perfectoid space with tile $X^\flat$, then 
\begin{enumerate}
  \item $X_\et\simeq X_\et^\flat$
  \item Let $X$ be affinoid perfectoid; $X=\spa(R,R^+)$. Then 
    \[
      \h^i(X_\et,\sO_X^+) = \begin{cases} R^+ & \text{if $i=0$} \\ \text{almost zero} & \text{if $i>0$} \end{cases}
    \]
\end{enumerate}
\end{proposition}

Part 1 follows from the fact that $|X|\simeq |X^\flat|$, combined with the almost 
purity theorem for finite \'etale covers. 

If $X=\spa(R,R^\circ)$ is affinoid of finite type over $K$, then 
$\h^i(X_\et,\sO_X)$ is $R$ for $i=0$, and is $0$ for $i>0$. On the other hand, 
$\h^i(X_\et,\sO_X^+)$ can contain lots of torsion, e.g. it could contain 
$K/ \cO_K$. This torsion gets killed on perfectoid covers. 

In Niziol's talks, we will see the following theorem. Let $K$ be algebraically 
closed of characteristic $0$, and let $X$ be a proper smooth variety over $K$. 
We can look at $\mathsf R\Gamma(X_\et,\sO_X^+)$. One has 
$\mathsf R\Gamma(X_\et,\sO_X^+)[\frac 1 p] = \mathsf R\Gamma(X_\et,\sO_X)$ 
(usual coherent cohomology). On the other hand, 
$\mathsf R\Gamma(X_\et,\sO_X^+)\lotimes_{\cO_K} \cO_K/p^n$ is almost 
isomorphic to $\mathsf R\Gamma(X_\et,\dZ/p^n)\lotimes_{\dZ/p^n} \cO_K / p^n$. 
This gives us a connection between cohomology $\sO_X^+$ on a perfectoid 
space, and more familiar \'etale cohomology. 

It is natural to ask whether there is a ``good'' category containing both 
rigid-analytic varieties and perfectoid spaces. In some sense, the answer is 
``yes'' because the category of adic spaces accomplishes this. On the other, 
hand, this is not a satisfactory answer because the category of adic spaces is 
not as nice as one might hope. The problem is that it is very hard to determine 
whether $\sO_X$ (or quasi-coherent modules) are sheaves.  

\begin{example}[Rost, Buzzard]
Consider $R=\dQ_p\langle p T, p T^{-1}\rangle$; the algebra of functions on the 
strip $p^{-1}\leqslant |T|\leqslant p$. Let $M$ be the Banach $R$-module with 
Banach $\dQ_p$-basis $\{p^{-n}\cdot T^{-n}, p^{-n} T^n:n\geqslant 0\}$. In 
other words, $M$ is the $p$-adic completion of the submodule of 
$\dQ_p\langle T^{\pm 1}\rangle$ generated by these elements. Let 
$R_1 = \dQ_p\langle p T,T^{-1}\rangle$, 
$R_2 = \dQ_p\langle T,p T^{-1}\rangle$. 
\end{example}

\begin{proposition}
The element $1\in M$ dies in $M\widehat\otimes_R R_i$ for $i=1,2$. In fact, 
$M\widehat\otimes_R R_i = 0$ for each $i$. 
\end{proposition}
\begin{proof}
We can assume $i=1$. Write $p^{-n} = (p^{-n} T^n)\cdot T^{-n} \in M\cdot R_1$. 
One has $|p^{-n} T^n|\leqslant 1$ and $|T^{-n}|\leqslant 1$. By the definition of 
the norm on $M\widehat\otimes_R R_i$, one gets 
$|p^{-n}|\leqslant 1$ for all $n$, i.e. 
$|1|\leqslant p^{-n}$ for all $n$, whence $|1|=0$, so $1$ dies. 
\end{proof}

\begin{theorem}
$R\oplus M$ ($M$ a square-zero ideal) violates the sheaf property. 
\end{theorem}
\begin{proof}In fact, there is an element which is globally non-zero, but is locally 
zero. Note that $R\oplus M$ is not spectral in the sense of Fontaine, because it 
has nilpotents. 
\end{proof}

\begin{proposition}
If $R$ is spectral, then for any cover $X=\spa(R,R^+) = \bigcup U_i$ with the 
$U_i$ rational, one has 
$R\hookrightarrow \prod\sO_X(U_i)$. In fact, 
$R\hookrightarrow \prod_{x\in X} \widehat{k(x)}$. 
\end{proposition}
\begin{proof}
This follows from a theorem of Berkovich to the effect that the pullback of the 
supremum norm on $\prod \widehat{k(x)}$ makes $R^\circ$ the $|\cdot|\leqslant 1$ 
subring. 
\end{proof}





\subsection{Open problems}

A natural question is: are there counterexamples to the sheaf property for 
spectral rings? Also, is there spectral $R$ with $U\subset \spa(R,R^+)$ such that 
$\sO_X(U)$ is not spectral? Finally, could these two phenomena occur simultaneously?

The following question is due to Rapoport: is ``perectoid'' a local property? Let 
$K$ be a perfectoid field. Let $X$ be a perfectoid space over $K$. Assume 
$X=\spa(R,R^+)$ is affinoid. Is $R$ perfectoid? We know that there is a covering 
of $X$ by rational subsets which are perfectoid, but it is not clear that this 
implies the ``perfectoidness'' of $X$. So we have to distinguish between 
``affinoid'' and ``perfectoid affinoid'' subsets of a perfectoid space. 

One can consider inverse limits. Let $(X_i)_{i\geqslant 0}$ be a 
tower of reduced adic spaces, all of finite type over a field $K$. Moreover, we 
assume that the transition maps are finite. Let $X$ be a perfectoid space, and 
$\{f_i:X\to X_i\}_{i\geqslant 0}$ a compatible system of maps. 

\begin{definition}
\begin{enumerate}
  \item Say that $X$ is a \emph{naive inverse limit} if all $X_i=\spa(R_i,R_i^+)$ 
    and $X=\spa(R,R^+)$, everything is affinoid perfectoid, and 
    $R^+$ is the $\pi$-adic completion of $\varinjlim R_i^+$. 
  \item Say $X\sim \varprojlim X_i$ if this is satisfied locally. 
\end{enumerate}
\end{definition}

The category of affinoid rings does not admit filtered direct limits. If 
$\{(R_i,R_i^+)\}_i$ is a direct system, what topology should we put on 
$\varinjlim R_i^+$? With the direct limit topology, $\varinjlim R_i^+$ is not 
affinoid, and it is not clear what other topology to impose. On the other hand, 
if the $R_i$ are spectral, we can let $R^+$ be the $\varpi$-adic completion of 
$\varinjlim R_i^+$ and $R=R^+[\frac 1 \varpi]$. 

\begin{proposition}
If $X\sim \varprojlim X_i$, then $X=\varprojlim X_i$ in the category of 
locally spectral adic spaces. 
\end{proposition}

Another question: does it make sense to develop the theory of spectral adic spaces? 
Spectral adic spaces are reduced, so they ``don't see'' tangent spaces, at least 
not via $K[\varepsilon]/\varepsilon^2$-valued points. For example, the isomorphism of 
Lubin-Tate and Drinfel'd towers, we have 
\[
  \varinjlim \mathcal M_i \sim \mathcal M_\infty\simeq \mathcal N_\infty \sim \varinjlim \mathcal N_i
\]
But is $\Omega_{\mathcal M_0}^1|_{\mathcal M_\infty}\simeq \Omega_{\mathcal N_0}^1|_{\mathcal N_\infty}$? This is not known. 





\subsection{Generic fibers of formal schemes}

This section is preparation for Weinstein's talk. Let $R$ be complete for the 
$I$-adic topology for some finitely generated ideal $I\subset R$. One can 
consider $\spa(R,R)$, which comes equipped with a structure-presheaf. Assume 
$R$ is a $\dZ_p$-algebra, and $p\in I$. Then $\spa(R,R)$ lies over 
$\spa(\dZ_p,\dZ_p) = \{s,\eta\}$, where $s$ is the special point and $\eta$ is 
the generic point. One has $\eta=\spa(\dQ_p,\dZ_p)$. 

\begin{definition}
The \emph{generic fiber} of $\spf R$ is $\spa(R,R)_\eta = \spa(R,R)\times_{\spa(\dZ_p,\dZ_p)} \spa(\dQ_p,\dZ_p)$. 
\end{definition}
The generic fiber is an open subset of $\spa(R,R)$. 

In any reasonable setup, $\spf R\mapsto \spa(R,R)$ is a fully faithful functor 
$\{\text{formal schemes}\}\hookrightarrow \{\text{adic spaces}\}$. 

\begin{example}
Suppose $R=\dZ_p\langle T\rangle$. Then 
\[
  \spa(R,R)_\eta = \spa(R[p^{-1}],R) = \spa(\dQ_p\langle T\rangle, \dZ_p\langle T\rangle) 
\]
the closed unit disk. 
\end{example}

\begin{example}
If $R=\dZ_p\pow T$ with $I=(p,T)$, then $\spa(R,R)_\eta$ is the open unit disk, 
hence is \emph{not} affinoid. 
\end{example}

\begin{example}
Let $R=\cO_K\pow{T^{1/p^\infty}}$, the $(p,T)$-adic completion of 
$\cO_K[T^{1/p^\infty}]$, where $K$ is some perfectoid field. Then 
$\spa(R,R)_\eta$ is the ``perfectoid open unit ball.'' 
\end{example}

\begin{proposition}
Fix $f_1,\dots,f_k\in I$ such that $I=(p,f_1,\dots,f_k)$. Then 
\[
  \spa(R,R)_\eta = \bigcup_{n\geqslant 1} \spa(R\langle \frac{f_i^n}{p},\dots,\frac{f_k^n}{p}\rangle[\frac 1 p])
\]
where we assume $|f_i|\leqslant |p|^{1/n}$. 
\end{proposition}

\begin{proposition}
Assume $R$ is a flat commutative $\cO_K$-algebra, where $K$ is a perfectoid field. 
Moreover, assume $\Phi:R/p^{1/p}\to R/p$ is an isomorphism. Then 
$\spa(R,R)_\eta$ is a perfectoid space over $K$. 
\end{proposition}

