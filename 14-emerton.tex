% !TEX root = msri731.tex

\section{Shimura varieties and perfectoid spaces 1: completed cohomology}
\thanksauthor{Matthew Emerton (Feb.\ 20)}





This is a report on Scholze's preprint \cite{sc13b}. 



\subsection{Some definitions}


Let's start with $G$, the $\dZ_p$-points of a some algebraic group. The group 
$G$ has a natural filtration $G_\bullet$, where $G_r$ is the level $p^r$ congruence 
subgroup of $G$. 

Let $X$ be a manifold with a tower $X_\bullet$ over $X$. We require that 
$G/G_r$ acts on each $X_r$, making it a principal homogeneous space over $X$. 

\begin{example}
We could have $G=\dZ_p$, and have $G$ act on the tower 
$\cdots \to S^1 \to S^1 \to \cdots \to S^1$, with each map being ``raise to 
$p$-th power.'' 
\end{example}

\begin{example}
Let $X=Y(N)$ be a modular curve, $X_r=Y(N p^r)$, and 
$G_r=\gl_2(\dZ/p^r)$. The group $G$ is just $\gl_2(\dZ_p)$. 
\end{example}

In this context, one can define the completed cohomology of the tower by 
\[
  \widetilde \h^i(X_\bullet,\dZ_p) = \varprojlim_s \varinjlim_r \h^i(X_r,\dZ/p^s)
\]
An important thing to notice is that these completed cohomology groups are 
$p$-adically complete, and ``see $p$-power torsion.'' Also, $p$-power torsion 
at the finite level can ``patch together'' to yield torsion-free cohomology 
classes at the infinite level. 

\begin{example}
Go back to $G=\dZ_p$ and $X_i=S^1$. Then $\widetilde h^0=\dZ_p$ and 
$\widetilde h^1=0$. 
\end{example}

\begin{example}
Here let $X_r=Y(N p^r)$ and $G=\gl_2(\dZ_p)$. One gets 
$\widetilde h^0=\dZ_p\pow{(\dZ/N)^\times \times \dZ_p^\times}$ and 
$\widetilde h^1$ is interesting in the context of $p$-adic Langlands. 
\end{example}

It is natural to ask about how the relation between this completed cohomology 
and cohomology at finite level. This is given by a Hochschild-Serre spectral 
sequence:
\[
  E_2^{i,j} = \h^i(G_r,\widetilde \h^j{}) \Rightarrow \h^{i+j}(X_r,\dZ_p) .
\]

For the tower of circles, $\h^0(\dZ_p,\dZ_p)=\h^1(\dZ_p,\dZ_p)=\dZ_p$. 
So some part of the cohomology comes from the cohomology of a $p$-adic Lie group, 
but that part ``dies'' when taking completed cohomology. 
This often lets one restrict to looking at ``Hecke eigenvalues on completed 
cohomology.'' 

If $W$ is a free $\dZ_p$-module of finite rank, with a continuous $G$-action, 
then we get compatible local systems $\mathscr W_r$ over each $X_r$. We have a 
spectral sequence 
\[
  E_2^{i j} = \ext_{\dZ_p\pow{G_r}}(W^\vee,\widetilde H^j{}) \Rightarrow \h^{i+j}(X_r,\mathscr W_r) .
\]
So somehow completed cohomology packages all the various choices of weight and 
level into a single object. 





\subsection{The setting}

Let $\dG$ be a reductive group over $\dQ$. Let $K_f$ be an open compact subgroup of 
$\dG(\dA^f)$. Write $Y(K_f)$ for the quotient 
\[
  Y(K_f) = \dG(\dQ) \backslash \dG(\dA) / A_\infty^\circ K_\infty^\circ K_f ,
\]
where $A_\infty^\circ$ is the connected component of $\dR$-points of the maximal 
$\dQ$-split torus in the center of $\dG$, and $K_\infty^\circ$ is the connected 
component of the maximal compact subgroup of $\dG(\dR)$. 

\begin{example}
Suppose $\dG=\gl(2)$. Then $A_\infty^\circ=\dR_{>0}^\times$ and $K_\infty^\circ=SO(2)$. 
Thus the quotient $\gl_2(\dR)/A_\infty^\circ K_\infty^\circ = \dC\smallsetminus\dR$. 
It turns out that $Y(K_f)$ is a modular curve of some type. 
\end{example}

Suppose $\dG$ is $\gl(2)$ over some imaginary quadratic field. Then 
$\dG(\dR)=\gl_2(\dC)$, $A_\infty^\circ=\dR_{>0}^\times$, 
$K_\infty^\circ=U(2)$, and 
\[
  \gl_2(\dC) / A_\infty^\circ K_\infty^\circ = PSL_2(\dC) / SO(3) = \mathbb H^3 , 
\]
which is hyperbolic 3-space. 

The manifolds $Y(K_f)$ are not generally algebraic varieties -- they are only 
manifolds. This makes it especially surprizing that one can attach Galois 
representations to torsion classes in their cohomology. 

\begin{theorem}[Franke]
$\h^i(Y(K_f),\dC)$ is ``computed by automorphic forms.'' 
\end{theorem}

This is a generalized Eichler-Shimura theory. 

Write $\dim Y(K_f)=2 q_0+\ell_0$, where 
\begin{align*}
  \ell_0 &= \operatorname{rk}(G) - \operatorname{rk}(A_\infty^\circ) - \operatorname{rk}(K_\infty^\circ) \\
  q_0 &= ?
\end{align*}

We know that for algebraic varieties, the ``most interesting cohomology'' occurs 
in ``middle degree.'' In some sense, $\h^{q_0}$ is the ``first interesting degree'' in the 
cohomology (with $\dC$-coefficients). 

Fix a ground level $K_f=K^p K_p$. (Here $K^p$ is a ``prime-to-$p$- part'' and 
$K_p$ is a ``$p$-part.'') We'll fix $K^p$ and vary $K_p$, to get a tower 
$Y(K^p K_{p^r})$. The group $G(\dQ_p)$ acts on this tower. Form 
$\widetilde\h^i$ and $\widetilde\h_c^i$; both of these admit an action of 
$G(\dQ_p)$. One would hope that these are described by $p$-adic local 
Langlands. Let $\dT$ be the algebra generated by ``Hecke operators of level 
$\ell\nmid p N$,'' where $N$ is the level of $K^p$. 


\begin{conjecture}[Calegari, Emerton]
$\widetilde \h^i{}=0$ if $i>q_0$. 
\end{conjecture}

\begin{theorem}[Scholze]
The conjecture is true for many Shimura varieties. 
\end{theorem}

[\ldots stopped writing on board, drew picture of $Y_0(11)$\ldots]





\subsection{Main example}

Let $\dG=U(2,2)$. Choose a quadratic imaginary field $F$, e.g. $\dQ(i)$. 
Let $V=F^{\oplus 4}$ with the Hermetian form $Q(x,y,z,w)=x\bar y-z \bar w$. 
The group $G(2,2)$ is the symmetries of the Hermetian form $Q$. The maximal 
torus of $\dG$ has real rank two. There are two maximal parabolics. 

The Klingon parabolic stabilizes the isotropic line. It has Levi subgroup 
isomorphic to $F^\times \times U(1,1)$. 

The Seigel parabolic stabilizes the isotropic plane. This has Levi subgroup 
equal to $\gl_2(F)$. 

One will have $Y\subset \bar Y\supset\partial = \bar Y\smallsetminus Y$. 
The boundary $\partial $ contains a $\partial_P$, which ``looks like'' a nil bundle 
over $Y_M$, where $M$ is a Levi for $P$. 

We have a sequence 
\[\xymatrix{
  \cdots \ar[r] 
    & \h_c^i(Y_\dG) \ar[r] \ar@{=}[d] 
    & \h^i(Y_\dG) \ar[r] \ar@{=}[d] 
    & \h^i(\partial) \ar[r] 
    & \cdots \\
  & \h^i(\bar Y_\dG,\partial) 
  & \h^i(\bar Y_\dG) 
  & \h_c^i(\partial_P) \ar[u] 
}\]
This let's reduce everything to $\h^i(\partial)$, and even better 
$\h^i(Y_P)$, or $\h_c^{i+1}(Y_P)$. From this long exact sequence, we see that 
to attach Galois representations to systems of Hecke eigenvalues appearing in 
$\h^i(\partial_M)$, it suffices to do so for systems of Hecke eigenvalues 
appearing in $\h^\bullet_c(Y_\dG)$. 

We can pass to the completed cohomology, getting a similar commutative 
diagram, and obtain $\widetilde \h_c^i(\partial_P)=\widetilde \h_c^i(\partial_M)$. 
Using Hochschild-Serre, it suffices to consider 
$\widetilde \h_c^\bullet(Y_\dG)$. For this, think of 
$\widetilde\h_c^\bullet(Y_\dG)$ as \'etale cohomology of a perfectoid space (a 
Shimura variety ``at infinite level'') and use a comparison theorem to compare 
with coherent cohomology. Ultimately, this reduces to classical modular forms 
on $U(2,2)$. Thanks to the work of many people, we already know how to 
associate Galois representations to holomorphic forms on $U(2,2)$. We can 
``chase these back'' to attach Galois representations to cohomology classes of 
hyperbolic three-manifolds. 

Of course, Scholze does this in far greater generality, i.e.\ for 
$U(n,n)$, or $\gl_n$ of a totally real or CM field. 








